\subsection{Incremental computation}
\label{sec:incremental}

For every prefix~$\Prefix$ evaluated during
Algorithm~\ref{alg:branch-and-bound}'s execution, we compute
the objective lower bound~${b(\Prefix, \x, \y)}$ and, sometimes,
the objective~${\Obj(\RL, \x, \y)}$ of the corresponding rule list~$\RL$.
%
These calculations are the dominant computations with respect to execution time.
%
This motivates our use of a highly optimized library,
designed by~\citet{YangRuSe16} for representing rule lists and
performing operations encountered in evaluating functions of rule lists.
%
Furthermore, we exploit the hierarchical nature of the objective
function and its lower bound to compute these quantities
incrementally throughout branch-and-bound execution.
%
\begin{arxiv}
In this section, we provide explicit expressions for
the incremental computations that are central to our approach.
\end{arxiv}
%
Later, in~\S\ref{sec:implementation}, we describe a cache data structure
for supporting our incremental framework in practice.

\begin{arxiv}
For completeness, before presenting our incremental expressions,
let us begin by writing down the objective lower bound and objective
of the empty rule list, ${\RL = ((), (), \Default, 0)}$,
the first rule list evaluated in Algorithm~\ref{alg:branch-and-bound}.
%
Since its prefix contains zero rules, it has zero prefix
misclassification error and also has length zero.
%
Thus, the empty rule list's objective lower bound is zero:
\begin{align}
  b((), \x, \y) = \Loss_p((), (), \x, \y) + \Reg \cdot 0 = 0.
\end{align}
%
Since none of the data are captured by the empty prefix, the default rule
corresponds to the majority class, and the objective corresponds to the
default rule misclassification error, \ie
\begin{align}
  \Obj(\RL, \x, \y) = \Loss(\RL, \x, \y) + \Reg \cdot 0
  &= \Loss_p((), (), \x, \y) + \Loss_0((), \Default, \x, \y) \nn \\
  &= b((), \x, \y) + \Loss_0((), \Default, \x, \y) = \Loss_0((), \Default, \x, \y).
\end{align}

Now, we derive our incremental expressions for the objective function and its lower bound.
%
Let ${\RL = (\Prefix, \Labels, \Default, K)}$ and
${\RL' = (\Prefix', \Labels', \Default', K + 1)}$
be rule lists such that prefix ${\Prefix = (p_1, \dots, p_K)}$
is the parent of ${\Prefix' = (p_1, \dots, p_K, p_{K+1})}$.
%
Let ${\Labels = (q_1, \dots, q_K)}$ and
${\Labels' = (q_1, \dots, q_K, q_{K+1})}$ be the corresponding labels.
%
The hierarchical structure of Algorithm~\ref{alg:branch-and-bound}
enforces that if we ever evaluate~$\RL'$, then we will have already
evaluated both the objective and objective lower bound of its parent,~$\RL$.
%
We would like to reuse as much of these computations as possible
in our evaluation of~$\RL'$.
%
We can write the objective lower bound of~$\RL'$ incrementally,
with respect to the objective lower bound of~$\RL$:
\begin{align}
b(\Prefix', \Labels', \x, \y)
  &= \Loss_p(\Prefix', \Labels', \x, \y) + \Reg (K + 1) \nn \\
&= \frac{1}{N} \sum_{n=1}^N \sum_{k=1}^{K+1} \Cap(x_n, p_k \given \Prefix')
  \wedge \one [ q_k \neq y_n ] + \Reg (K+1) \label{eq:non-inc-lb} \\
&= \Loss_p(\Prefix, \Labels, \x, \y) + \Reg K + \Reg
  + \frac{1}{N} \sum_{n=1}^N \Cap(x_n, p_{K+1} \given \Prefix') \wedge \one [q_{K+1} \neq y_n ] \nn \\
&= b(\Prefix, \Labels, \x, \y) + \Reg
  + \frac{1}{N} \sum_{n=1}^N \Cap(x_n, p_{K+1} \given \Prefix') \wedge \one [q_{K+1} \neq y_n ] \nn \\
&= b(\Prefix, \Labels, \x, \y) + \Reg  + \frac{1}{N} \sum_{n=1}^N \neg\, \Cap(x_n, \Prefix) \wedge
  \Cap(x_n, p_{K+1}) \wedge \one [q_{K+1} \neq y_n].
\label{eq:inc-lb}
\end{align}
%Notice that the term~${\neg\, \Cap(x_n, \Prefix)}$ in~\eqref{eq:inc-lb}
%depends only on~$\Prefix$, \ie it does not depend on~$\Prefix'$; furthermore,
%it is computed in the evaluation of~$\RL$'s default rule misclassification error,
%\begin{align}
%\Loss_0(\Prefix, \Default, \x, \y) = \frac{1}{N}\sum_{n=1}^N \neg\, \Cap(x_n, \Prefix) \wedge \one [q_0 \neq y_n].
%\end{align}
Thus, if we store $b(\Prefix, \Labels, \x, \y)$, % and ${\neg\, \Cap(\x, \Prefix)}$,
then we can reuse this quantity when computing $b(\Prefix', \Labels', \x, \y)$.
%
Transforming~\eqref{eq:non-inc-lb} into~\eqref{eq:inc-lb} yields a
significantly simpler expression that is a function of the stored
quantity~$b(\Prefix, \Labels, \x, \y)$. %, as well as ${\neg\, \Cap(\x, \Prefix)}$
%and the last rule of~$\RL'$, ${p_{K+1} \rightarrow q_{K+1}}$.
%
For the objective of~$\RL'$, first let us write a na\"ive expression:
\begin{align}
\Obj(\RL', \x, \y) &= \Loss(\RL', \x, \y) + \Reg (K + 1)
= \Loss_p(\Prefix', \Labels', \x, \y) + \Loss_0(\Prefix', \Default', \x, \y) + \Reg(K + 1) \nn \\
&= \frac{1}{N} \sum_{n=1}^N \sum_{k=1}^{K+1} \Cap(x_n, p_k \given \Prefix')
  \wedge \one [ q_k \neq y_n ] + \frac{1}{N}\sum_{n=1}^N \neg\, \Cap(x_n, \Prefix') \wedge
  \one [q'_0 \neq y_n] + \Reg (K+1). \label{eq:non-inc-obj}
\end{align}
Instead, we can compute the objective of~$\RL'$ incrementally
with respect to its objective lower bound:
\begin{align}
\Obj(\RL', \x, \y) &=  \Loss_p(\Prefix', \Labels', \x, \y) +
  \Loss_0(\Prefix', \Default', \x, \y) + \Reg (K + 1) \nn \\
&= b(\Prefix', \Labels', \x, \y) + \Loss_0(\Prefix', \Default', \x, y) \nn \\
&= b(\Prefix', \Labels', \x, \y) + \frac{1}{N}\sum_{n=1}^N \neg\, \Cap(x_n, \Prefix') \wedge
  \one [q'_0 \neq y_n] \nn \\
&= b(\Prefix', \Labels', \x, \y) + \frac{1}{N}\sum_{n=1}^N \neg\, \Cap(x_n, \Prefix) \wedge
  (\neg\, \Cap(x_n, p_{K+1})) \wedge \one [q'_0 \neq y_n].
\label{eq:inc-obj}
\end{align}
The expression in~\eqref{eq:inc-obj} is much simpler than the na\"ive
one in~\eqref{eq:non-inc-obj}, and is a function of
$b(\Prefix', \Labels', \x, \y)$, which we computed in~\eqref{eq:inc-lb}.
%as well as ${\neg\, \Cap(\x, \Prefix)}$,
%and the last antecedent~$p_{K+1}$ and default rule~$\Default'$ of~$\RL'$.
Though we could compute the objective of~$\RL'$ incrementally
with respect to that of~$\RL$, doing so would in practice require
that we also store~$\Obj(\RL, \x, \y)$; we prefer the approach suggested
by~\eqref{eq:inc-obj} since it avoids this additional storage overhead.

\begin{algorithm}[t!]
  \caption{Incremental branch-and-bound for learning rule lists, for simplicity, from a cold start.
  We explicitly show the incremental objective lower bound and objective functions in Algorithm~\ref{alg:incremental-functions}.}
\label{alg:incremental}
\begin{algorithmic}
\normalsize
\State \textbf{Input:} Objective function~$\Obj(\RL, \x, \y)$,
objective lower bound~${b(\Prefix, \x, \y)}$,
set of antecedents ${\RuleSet = \{s_m\}_{m=1}^M}$,
training data~$(\x, \y) = {\{(x_n, y_n)\}_{n=1}^N}$,
regularization parameter~$\Reg$
\State \textbf{Output:} Provably optimal rule list~$\OptimalRL$ with minimum objective~$\OptimalObj$ \\

\State $\CurrentRL \gets ((), (), \Default, 0)$ \Comment{Initialize current best rule list with empty rule list}
\State $\CurrentObj \gets \Obj(\CurrentRL, \x, \y)$ \Comment{Initialize current best objective}
\State $Q \gets $ queue$(\,[\,(\,)\,]\,)$ \Comment{Initialize queue with empty prefix}
\State $C \gets $ cache$(\,[\,(\,(\,)\,, 0\,)\,]\,)$ \Comment{Initialize cache with empty prefix and its objective lower bound}
\While {$Q$ not empty} \Comment{Optimization complete when the queue is empty}
	\State $\Prefix \gets Q$.pop(\,) \Comment{Remove a prefix~$\Prefix$ from the queue}
        \State $b(\Prefix, \x, \y) \gets C$.find$(\Prefix)$ \Comment{Look up $\Prefix$'s lower bound in the cache}
        \State $\mathbf{u} \gets \neg\,\Cap(\x, \Prefix)$ \Comment{Bit vector indicating data not captured by $\Prefix$}
        \For {$s$ in $\RuleSet$} \Comment{Evaluate all of $\Prefix$'s children}
            \If {$s$ not in $\Prefix$}
                \State $\PrefixB \gets (\Prefix, s)$ \Comment{\textbf{Branch}: Generate child $\PrefixB$}
                \State $\mathbf{v} \gets \mathbf{u} \wedge \Cap(\x, s)$ \Comment{Bit vector indicating data captured by $s$ in $\PrefixB$}
                \State $b(\PrefixB, \x, \y) \gets b(\Prefix, \x, \y) + \Reg~ + $ \Call{IncrementalLowerBound}{$\mathbf{v}, \y, N$} \Comment{Eq.~\eqref{eq:inc-lb}}
                \If {$b(\PrefixB, \x, \y) < \CurrentObj$} \Comment{\textbf{Bound}: Apply bound from Theorem~\ref{thm:bound}}
                    \State $\Obj(\RLB, \x, \y) \gets b(\PrefixB, \x, \y)~ + $ \Call{IncrementalObjective}{$\mathbf{u}, \mathbf{v}, \y, N$} \Comment{Eq.~\eqref{eq:inc-obj}}
                    \If {$\Obj(\RLB, \x, \y) < \CurrentObj$}
                        \State $(\CurrentRL, \CurrentObj) \gets (\RLB, \Obj(\RLB, \x, \y))$ \Comment{Update current best rule list and objective}
                    \EndIf
                    \State $Q$.push$(\PrefixB)$ \Comment{Add $\PrefixB$ to the queue}
                    \State $C$.insert$(\PrefixB, b(\PrefixB, \x, \y))$ \Comment{Add $\PrefixB$ and its lower bound to the cache}
                \EndIf
            \EndIf
        \EndFor
\EndWhile
\State $(\OptimalRL, \OptimalObj) \gets (\CurrentRL, \CurrentObj)$ \Comment{Identify provably optimal rule list and objective}
\end{algorithmic}
\end{algorithm}

\begin{algorithm}[t!]
  \caption{Incremental objective lower bound~\eqref{eq:inc-lb} used in Algorithm~\ref{alg:incremental}.}
\label{alg:incremental-lb}
\begin{algorithmic}
\normalsize
\State \textbf{Input:}
Bit vector~${\mathbf{v} \in \{0, 1\}^N}$ indicating data captured by $s$, the last antecedent in~$\PrefixB$,
bit vector of class labels~${\y \in \{0, 1\}^N}$,
number of observations~$N$
\State \textbf{Output:} Component of~$\RLB$'s misclassification error due to data captured by~$s$ \\

\Function{IncrementalLowerBound}{$\mathbf{v}, \y, N$}
    \State $n_v = \Count(\mathbf{v})$ \Comment{Number of data captured by $s$, the last antecedent in $\PrefixB$}
    \State $\mathbf{w} \gets \mathbf{v} \wedge \y$ \Comment{Bit vector indicating data captured by $s$ with label $1$}
    \State $n_w = \Count(\mathbf{w})$ \Comment{Number of data captured by $s$ with label $1$}
    \If {$n_w / n_v > 0.5$}
        \State \Return $(n_v - n_w) / N$ \Comment{Misclassification error of the rule $s \rightarrow 1$}
    \Else
        \State \Return $n_w / N$ \Comment{Misclassification error of the rule $s \rightarrow 0$}
    \EndIf
    \EndFunction
\end{algorithmic}
\end{algorithm}

\begin{algorithm}[t!]
  \caption{Incremental objective function~\eqref{eq:inc-obj} used in Algorithm~\ref{alg:incremental}.}
\label{alg:incremental-obj}
\begin{algorithmic}
\normalsize
\State \textbf{Input:}
Bit vector~${\mathbf{u} \in \{0, 1\}^N}$ indicating data not captured by~$\PrefixB$'s parent prefix,
bit vector~${\mathbf{v} \in \{0, 1\}^N}$ indicating data not captured by $s$, the last antecedent in~$\PrefixB$,
bit vector of class labels~${\y \in \{0, 1\}^N}$,
number of observations~$N$
\State \textbf{Output:} Component of~$\RLB$'s misclassification error due to its default rule \\

 \Function{IncrementalObjective}{$\mathbf{u}, \mathbf{v}, \y, N$}
    \State $\mathbf{f} \gets \mathbf{u} \wedge \neg\,\mathbf{v} $ \Comment{Bit vector indicating data not captured by $\PrefixB$}
    \State $n_f = \Count(\mathbf{f})$ \Comment{Number of data not captured by $\PrefixB$}
    \State $\mathbf{g} \gets \mathbf{f} \wedge \y$ \Comment{Bit vector indicating data not captured by $\PrefixB$ with label $1$}
    \State $n_g = \Count(\mathbf{w})$ \Comment{Number of data not captued by $\PrefixB$ with label $1$}
    \If {$n_f / n_g > 0.5$}
        \State \Return $(n_f - n_g) / N$ \Comment{Default rule misclassification error with label $1$}
    \Else
        \State \Return $n_g / N$ \Comment{Default rule misclassification error with label $0$}
    \EndIf
\EndFunction
\end{algorithmic}
\end{algorithm}

We present an incremental branch-and-bound procedure in
Algorithm~\ref{alg:incremental}, and show the incremental computations
of the objective lower bound~\eqref{eq:inc-lb} and objective~\eqref{eq:inc-obj}
as two separate functions in Algorithms~\ref{alg:incremental-lb}
and~\ref{alg:incremental-obj}, respectively.
%
In Algorithm~\ref{alg:incremental}, we use a cache to store
prefixes and their objective lower bounds.
%
Algorithm~\ref{alg:incremental} additionally reorganizes the structure
of Algorithm~\ref{alg:branch-and-bound} to group together the computations
associated with all children of a particular prefix.
%
This has two advantages.
%
The first is to consolidate cache queries: all children of the same
parent prefix compute their objective lower bounds with respect to
the parent's stored value, and we only require one cache `find' operation
for the entire group of children, instead of a separate query for each child.
%
The second is to shrink the queue's size:
instead of adding all of a prefix's children as separate queue elements,
we represent the entire group of children in the queue by a single element.
%
Since the number of children associated with each prefix
is close to the total number of possible antecedents,
both of these effects can yield significant savings.
%
For example, if we are trying to optimize over rule lists formed
from a set of 1000 antecedents, then the maximum queue size in
Algorithm~\ref{alg:incremental} will be smaller than that in
Algorithm~\ref{alg:branch-and-bound} by a factor of nearly 1000.

\end{arxiv}
