%\section{Lower bounds for proving optimality}
%
%Our problem has rich structure beyond the hierarchical objective lower bound
%in Theorem~\ref{thm:bound}, including several instances of symmetries.
%%
%In this section, we enumerate a series of additional bounds and
%symmetries that combine to yield aggressive pruning opportunities
%throughout the execution of our branch-and-bound algorithm.
%%
%We additionally quantify the computational consequences of many
%of these observations.
%%
%Below, we begin with an immediate consequence of Theorem~\ref{thm:bound}.
%%followed by a more general version of Theorem~\ref{thm:bound}.

Next, we state an immediate consequence of Theorem~\ref{thm:bound}.

\begin{lemma}[Objective lower bound with one-step lookahead]
\label{lemma:lookahead}
Let~$\Prefix$ be a $K$-prefix
and let~$\CurrentObj$ be the current best objective.
%
If ${b(\Prefix, \x, \y) + \Reg \ge \CurrentObj}$,
then for any $K'$-rule list ${\RL' \in \StartsWith(\Prefix)}$
whose prefix~$\Prefix'$ starts with~$\Prefix$ and~${K' > K}$,
it follows that ${\Obj(\RL', \x, \y) \ge \CurrentObj}$.
\end{lemma}

\begin{arxiv}
\begin{proof}
By the definition of the lower bound~\eqref{eq:lower-bound},
which includes the penalty for longer prefixes,
\begin{align}
\Obj(\Prefix', \x, y) \ge b(\Prefix', \x, \y) &= \Loss_p(\Prefix', \x, \y) + \Reg K' \nn \\
&= \Loss_p(\Prefix', \x, \y) + \Reg K + \Reg (K' - K) \nn \\
&= b(\Prefix, \x, \y) + \Reg (K' - K)
\ge b(\Prefix, \x, \y) + \Reg \ge \CurrentObj.
\label{eq:lookahead}
\end{align}
\end{proof}
\end{arxiv}

Therefore, even if we encounter a prefix~$\Prefix$
with lower bound ${b(\Prefix, \x, \y) \le \CurrentObj}$,
if ${b(\Prefix, \x, \y) + \Reg \ge \CurrentObj}$, then we can prune
all prefixes~$\Prefix'$ that start with and are longer than~$\Prefix$.

%Before proceeding, let us summarize the remainder of this section.
%%
%First, we prove upper bounds on prefix
%length~(\S\ref{sec:ub-prefix-length}),
%and corresponding upper bounds on the number of
%prefix evaluations~(\S\ref{sec:ub-size}).
%%
%\begin{arxiv}
%Next, we prove lower~(\S\ref{sec:lb-support})
%and upper~(\S\ref{sec:ub-support}) bounds on antecedent support,
%and provide a unified analysis of consequences of
%these antecedent support bounds~(\S\ref{sec:reject}).
%\end{arxiv}
%\begin{kdd}
%Next, we prove lower~(\S\ref{sec:lb-support}) bounds on
%antecedent support, and describe algorithmic consequences.
%\end{kdd}
%%
%We then analyze rule lists whose prefixes
%capture the same data~(\S\ref{sec:equivalent}),
%which enables symmetry-aware garbage collection.
%%
%In particular, we highlight the special case of prefixes
%that are equivalent up to a permutation of their
%antecedents~(\S\ref{sec:permutation});
%this enables permutation-aware garbage collection,
%which we characterize via another upper bound on
%the number of prefix evaluations~(\S\ref{sec:permutation-counting}).
%%
%\begin{arxiv}
%Next, we slightly generalize the result from~\S\ref{sec:equivalent}
%to rule lists whose prefixes capture nearly the same data~(\S\ref{sec:similar}).
%%
%\dots
%\end{arxiv}
%%
%Finally, we show how to tighten the objective lower bound,
%sometimes dramatically, when the data of interest partitions
%into non-singleton equivalence classes~(\S\ref{sec:identical}).

\subsection{Upper bounds on prefix length}
\label{sec:ub-prefix-length}

\begin{arxiv}
The simplest upper bound on prefix length is given by the total
number of available antecedents.

\begin{proposition}[Trivial upper bound on prefix length]
\label{prop:trivial-length}
Consider a state space of all rule lists formed from
a set of~$M$ antecedents,
and let~$L(\RL)$ be the length of rule list~$\RL$.
%
$M$ provides an upper bound on the length of
any optimal rule list
${\OptimalRL \in \argmin_\RL \Obj(\RL, \x, \y)}$,
\ie ${L(\RL) \le M}$.
\end{proposition}

\begin{proof}
Rule lists consist of distinct rules by definition.
\end{proof}
\end{arxiv}

At any point during branch-and-bound execution, the current best objective~$\CurrentObj$
implies an upper bound on the maximum prefix length we might still have to consider.
%
\begin{theorem}[Upper bound on prefix length]
\label{thm:ub-prefix-length}
Consider a state space of all rule lists formed from a set of~$M$ antecedents.
%
Let~$L(\RL)$ be the length of rule list~$\RL$
and let~$\CurrentObj$ be the current best objective.
%
For all optimal rule lists ${\OptimalRL \in \argmin_\RL \Obj(\RL, \x, \y)}$
\begin{arxiv}
\begin{align}
L(\OptimalRL) \le \min \left(\left\lfloor \frac{\CurrentObj}{\Reg} \right\rfloor, M \right),
\label{eq:max-length}
\end{align}
\end{arxiv}
\begin{kdd}
\begin{align}
L(\OptimalRL) \le \min \left(\left\lfloor \CurrentObj / \Reg \right\rfloor, M \right),
\label{eq:max-length}
\end{align}
\end{kdd}
where~$\Reg$ is the regularization parameter.
%
\begin{arxiv}
Furthermore, if the current best rule list~$\CurrentRL$
has length~$K$ and zero misclassification error,
then for every optimal rule list
${\OptimalRL \in \argmin_\RL \Obj(\RL, \x, \y)}$,
if ${\CurrentRL \in}$ ${\argmin_d \Obj(\RL, \x, \y)}$,
then ${L(\OptimalRL) \le K}$,
or otherwise if ${\CurrentRL \notin}$ ${\argmin_d \Obj(\RL, \x, \y)}$,
then ${L(\OptimalRL) \le K - 1}$.
\end{arxiv}
\end{theorem}

\begin{arxiv}
\begin{proof}
For an optimal rule list~$\OptimalRL$ with objective~$\OptimalObj$,
\begin{align}
\Reg L(\OptimalRL) \le \OptimalObj = \Obj(\OptimalRL, \x, \y)
= \Loss(\OptimalRL, \x, \y) + \Reg L(\OptimalRL)
\le \CurrentObj.
\end{align}
The maximum possible length for~$\OptimalRL$ occurs
when~$\Loss(\OptimalRL, \x, \y)$ is minimized;
combining with Proposition~\ref{prop:trivial-length}
gives bound~\eqref{eq:max-length}.

For the rest of the proof,
let~${K^* = L(\OptimalRL)}$ be the length of~$\OptimalRL$.
%
If the current best rule list~$\CurrentRL$ has zero
misclassification error, then
\begin{align}
\Reg K^* \leq \Loss(\OptimalRL, \x, \y) + \Reg K^* = \Obj(\OptimalRL, \x, \y)
\le \CurrentObj = \Obj(\CurrentRL, \x, \y) = \Reg K,
\end{align}
and thus ${K^* \leq K}$.
%
If the current best rule list is suboptimal,
\ie ${\CurrentRL \notin \argmin_\RL \Obj(\RL, \x, \y)}$, then
%
\begin{align}
\Reg K^* \leq \Loss(\OptimalRL, \x, \y) + \Reg K^* = \Obj(\OptimalRL, \x, \y)
< \CurrentObj = \Obj(\CurrentRL, \x, \y) = \Reg K,
\end{align}
in which case ${K^* < K}$, \ie ${K^* \leq K-1}$, since $K$ is an integer.
\end{proof}

The latter part of Theorem~\ref{thm:ub-prefix-length} tells us that
if we only need to identify a single instance of an optimal rule list
${\OptimalRL \in \argmin_\RL \Obj(\RL, \x, \y)}$, and we encounter a perfect
$K$-prefix with zero misclassification error, then we can prune all
prefixes of length~$K$ or greater.

\end{arxiv}

\begin{corollary}[Simple upper bound on prefix length]
\label{cor:ub-prefix-length}
\begin{arxiv}
Let~$L(\RL)$ be the length of rule list~$\RL$.
\end{arxiv}
%
For all optimal rule lists ${\OptimalRL \in \argmin_\RL \Obj(\RL, \x, \y)}$,
\begin{arxiv}
\begin{align}
L(\OptimalRL) \le \min \left( \left\lfloor \frac{1}{2\Reg} \right\rfloor, M \right).
\label{eq:max-length-trivial}
\end{align}
\end{arxiv}
\begin{kdd}
\begin{align}
L(\OptimalRL) \le \min \left( \left\lfloor 1 / 2\Reg \right\rfloor, M \right).
\label{eq:max-length-trivial}
\end{align}
\end{kdd}
\end{corollary}

\begin{arxiv}
\begin{proof}
Let ${d = ((), (), q_0, 0)}$ be the empty rule list;
it has objective ${\Obj(\RL, \x, \y) = \Loss(\RL, \x, \y) \le 1/2}$,
which gives an upper bound on~$\CurrentObj$.
%
Combining with~\eqref{eq:max-length}
and Proposition~\ref{prop:trivial-length}
gives~\eqref{eq:max-length-trivial}.
\end{proof}
\end{arxiv}

For any particular prefix~$\Prefix$, we can obtain potentially tighter
upper bounds on prefix length for
\begin{arxiv}
the family of
\end{arxiv}
all prefixes that start with~$\Prefix$.

%
%\begin{kdd}
%By considering this constraint in the context of a specific
%prefix~$\Prefix$, we can obtain potentially tighter upper bounds on
%prefix length for the family of all prefixes that start with~$\Prefix$.
%\end{kdd}

\begin{theorem}[Prefix-specific upper bound on prefix length]
\label{thm:ub-prefix-specific}
Let ${\RL = (\Prefix, \Labels, \Default, K)}$ be a rule list, let
${\RL' = (\Prefix', \Labels', \Default', K') \in \StartsWith(\Prefix)}$
be any rule list such that~$\Prefix'$ starts with~$\Prefix$,
and let~$\CurrentObj$ be the current best objective.
%
If~$\Prefix'$ has lower bound~${b(\Prefix', \x, \y) < \CurrentObj}$, then
\begin{align}
K' < \min \left( K + \left\lfloor \frac{\CurrentObj - b(\Prefix, \x, \y)}{\Reg} \right\rfloor, M \right).
\label{eq:max-length-prefix}
\end{align}
\end{theorem}

\begin{arxiv}
\begin{proof}
First, note that~${K' \ge K}$, since~$\Prefix'$ starts with~$\Prefix$.
%
Now recall from~\eqref{eq:prefix-lb} that
%
\begin{align}
b(\Prefix, \x, \y) = \Loss(\RL, \Labels, \x, \y) + \Reg K
\le \Loss(\RL', \Labels', \x, \y) + \Reg K' = b(\Prefix', \x, \y),
\end{align}
%
and from~\eqref{eq:prefix-loss} that
${\Loss(\RL, \Labels, \x, \y) \le \Loss(\RL', \Labels', \x, \y)}$.
%
Combining these bounds and rearranging gives
\begin{align}
b(\Prefix, \x, \y) + \Reg (K' - K) \le b(\Prefix', \x, \y).
\label{eq:length-diff}
\end{align}
Combining~\eqref{eq:length-diff} with~${b(\Prefix', \x, \y) < \CurrentObj}$
and Proposition~\ref{prop:trivial-length} gives~\eqref{eq:max-length-prefix}.
\end{proof}
\end{arxiv}

We can view Theorem~\ref{thm:ub-prefix-specific} as a generalization
of our one-step lookahead bound (Lemma~\ref{lemma:lookahead});
rearranging to obtain a bound on ${K' - K}$
gives an upper bound on the number of remaining `steps' corresponding to
\begin{arxiv}
an iterative
\end{arxiv}
\begin{kdd}
a
\end{kdd}
sequence of single-rule extensions of a prefix~$\Prefix$.
%
\begin{arxiv}
Notice that when~${\RL = ((), (), q_0, 0)}$ is the empty rule list,
this bound replicates~\eqref{eq:max-length}, since~${b(\Prefix, \x, \y) = 0}$.
\end{arxiv}

\begin{arxiv}
\subsection{Upper bounds on the number of prefix evaluations}
\end{arxiv}
\begin{kdd}
\subsection{Upper bounds on prefix evaluations}
\end{kdd}
\label{sec:ub-size}

\begin{arxiv}
In this section, we use our upper bounds on prefix length
from~\S\ref{sec:ub-prefix-length} to derive corresponding
upper bounds on the number of prefix evaluations made by
Algorithm~\ref{alg:branch-and-bound}.
%
First, we
\end{arxiv}
\begin{kdd}
In this section, we use Theorem~\ref{thm:ub-prefix-specific}'s
upper bound on prefix length to derive a corresponding
upper bound on the number of prefix evaluations made by
Algorithm~\ref{alg:branch-and-bound}.
%
We
\end{kdd}
present Theorem~\ref{thm:remaining-eval-fine},
in which we use information about the state of
Algorithm~\ref{alg:branch-and-bound}'s execution
to calculate, for any given execution state,
upper bounds on the number of additional prefix evaluations that might
be required for the execution to complete.
%
This number of remaining evaluations is equal to the number of
prefixes that are currently in or will be inserted into the queue.
%
The relevant execution state depends on the current
best objective~$\CurrentObj$ and information about
prefixes we are planning to evaluate, \ie prefixes in the
queue~$\Queue$ of Algorithm~\ref{alg:branch-and-bound}.
%
\begin{arxiv}
After~Theorem~\ref{thm:remaining-eval-fine}, we present two
weaker propositions that provide useful intuition.
\end{arxiv}

\begin{arxiv}
\begin{theorem}[Fine-grain upper bound on the number of remaining prefix evaluations]
\end{arxiv}
\begin{kdd}
\begin{theorem}[Upper bound on the number of remaining prefix evaluations]
\end{kdd}
\label{thm:remaining-eval-fine}
Consider a state space of all rule lists formed from a set of~$M$ antecedents,
and consider Algorithm~\ref{alg:branch-and-bound} at a particular instant
during execution.
%
Let~$\CurrentObj$ be the current best objective, let~$\Queue$ be the queue,
and let~$L(\Prefix)$ be the length of prefix~$\Prefix$.
%
Define~${\Remaining(\CurrentObj, \Queue)}$ to be the number of remaining
prefix evaluations, then
\begin{align}
\Remaining(\CurrentObj, \Queue)
\le \sum_{\Prefix \in Q} \sum_{k=0}^{f(\Prefix)} \frac{(M - L(\Prefix))!}{(M - L(\Prefix) - k)!},
\end{align}
\begin{arxiv}
where
\begin{align}
f(\Prefix) = \min \left( \left\lfloor
  \frac{\CurrentObj - b(\Prefix, \x, \y)}{\Reg} \right\rfloor, M - L(\Prefix)\right).
\label{eq:f}
\end{align}
\end{arxiv}
\begin{kdd}
\begin{align}
\text{where} \quad f(\Prefix) = \min \left( \left\lfloor
  \frac{\CurrentObj - b(\Prefix, \x, \y)}{\Reg} \right\rfloor, M - L(\Prefix)\right).
\label{eq:f}
\end{align}
\end{kdd}
\end{theorem}

\begin{proof}
The number of remaining prefix evaluations is equal to the number of
prefixes that are currently in or will be inserted into queue~$\Queue$.
%
For any such prefix~$\Prefix$, Theorem~\ref{thm:ub-prefix-specific}
gives an upper bound on the length of any prefix~$\Prefix'$
that starts with~$\Prefix$:
\begin{align}
L(\Prefix') \le \min \left( L(\Prefix) + \left\lfloor \frac{\CurrentObj - b(\Prefix, \x, \y)}{\Reg} \right\rfloor, M \right)
\equiv U(\Prefix).
\end{align}
This gives an upper bound on the number of remaining prefix evaluations:
\begin{arxiv}
\begin{align}
\Remaining(\CurrentObj, \Queue)
\le \sum_{\Prefix \in Q} \sum_{k=0}^{U(\Prefix) - L(\Prefix)} P(M - L(\Prefix), k)
= \sum_{\Prefix \in Q} \sum_{k=0}^{f(\Prefix)} \frac{(M - L(\Prefix))!}{(M - L(\Prefix) - k)!}.
\end{align}
\end{arxiv}
\begin{kdd}
\begin{align}
\Remaining(\CurrentObj, \Queue)
\le \sum_{\Prefix \in Q} \sum_{k=0}^{U(\Prefix) - L(\Prefix)} P(M - L(\Prefix), k).
\end{align}
\end{kdd}
\end{proof}

\begin{arxiv}
Our first
\end{arxiv}
\begin{kdd}
The
\end{kdd}
proposition below is a na\"ive upper bound on
the total number of prefix evaluations over the course of
Algorithm~\ref{alg:branch-and-bound}'s execution.
%
It only depends on the number of rules and
the regularization parameter~$\Reg$;
\ie unlike Theorem~\ref{thm:remaining-eval-fine},
it does not utilize algorithm execution state to
bound the size of the search space.

\begin{proposition}[Upper bound on the total number of prefix evaluations]
\label{thm:ub-total-eval}
Define~$\TotalRemaining(\RuleSet)$ to be the total number of prefixes
evaluated by Algorithm~\ref{alg:branch-and-bound}, given the state space of
all rule lists formed from a set~$\RuleSet$ of~$M$ rules.
%
For any set~$\RuleSet$ of $M$ rules,
\begin{align}
\TotalRemaining(\RuleSet) \le \sum_{k=0}^K \frac{M!}{(M - k)!},
\label{eq:size-naive}
\end{align}
where ${K = \min(\lfloor 1/2 \Reg \rfloor, M)}$.
\end{proposition}

\begin{proof}
By Corollary~\ref{cor:ub-prefix-length},
${K \equiv \min(\lfloor 1 / 2 \Reg \rfloor, M)}$
gives an upper bound on the length of any optimal rule list.
%
\begin{arxiv}
Since we can think of our problem as finding the optimal
selection and permutation of~$k$ out of~$M$ rules,
over all~${k \le K}$,
\begin{align}
\TotalRemaining(\RuleSet) \le 1 + \sum_{k=1}^K P(M, k)
= \sum_{k=0}^K \frac{M!}{(M - k)!}.
\end{align}
\end{arxiv}
\begin{kdd}
We obtain~\eqref{eq:size-naive} by viewing
our problem as finding the optimal
selection and permutation of~$k$ out of~$M$ rules,
over all~${k \le K}$.
\end{kdd}
\end{proof}

\begin{arxiv}

Our next upper bound is strictly tighter than the bound in
Proposition~\ref{thm:ub-total-eval}.
%
Like Theorem~\ref{thm:remaining-eval-fine}, it uses the
current best objective and information about
the lengths of prefixes in the queue to constrain
the lengths of prefixes in the remaining search space.
%
However, Proposition~\ref{prop:remaining-eval-coarse}
is weaker than Theorem~\ref{thm:remaining-eval-fine} because
it leverages only coarse-grain information from the queue.
%
Specifically, Theorem~\ref{thm:remaining-eval-fine} is
strictly tighter because it additionally incorporates
prefix-specific objective lower bound information from
prefixes in the queue, which further constrains
the lengths of prefixes in the remaining search space.

\begin{proposition}[Coarse-grain upper bound on the number of remaining prefix evaluations]
\label{prop:remaining-eval-coarse}
Consider a state space of all rule lists formed from a set of~$M$ antecedents,
and consider Algorithm~\ref{alg:branch-and-bound} at a particular instant
during execution.
%
Let~$\CurrentObj$ be the current best objective, let~$\Queue$ be the queue,
and let~$L(\Prefix)$ be the length of prefix~$\Prefix$.
%
Let~$\Queue_j$ be the number of prefixes of length~$j$ in~$\Queue$,
\begin{align}
\Queue_j = \big | \{ \Prefix : L(\Prefix) = j, \Prefix \in \Queue \} \big |
\end{align}
and let~${J = \argmax_{\Prefix \in \Queue} L(\Prefix)}$
be the length of the longest prefix in~$\Queue$.
%
Define~${\Remaining(\CurrentObj, \Queue)}$ to be the number of remaining
prefix evaluations, then
\begin{align}
\Remaining(\CurrentObj, \Queue)
\le \sum_{j=1}^J \Queue_j \left( \sum_{k=0}^{K-j} \frac{(M-j)!}{(M-j - k)!} \right),
\end{align}
where~${K = \min(\lfloor \CurrentObj / \Reg \rfloor, M)}$.
\end{proposition}

\begin{proof}
The number of remaining prefix evaluations is equal to the number of
prefixes that are currently in or will be inserted into queue~$\Queue$.
%
For any such remaining prefix~$\Prefix$,
Theorem~\ref{thm:ub-prefix-length} gives an upper bound on its length;
define~$K$ to be this bound:
${L(\Prefix) \le \min(\lfloor \CurrentObj / \Reg \rfloor, M) \equiv K}$.
%
For any prefix~$\Prefix$ in queue~$\Queue$ with length~${L(\Prefix) = j}$,
the maximum number of prefixes that start with~$\Prefix$
and remain to be evaluated is:
\begin{align}
\sum_{k=0}^{K-j} P(M-j, k) = \sum_{k=0}^{K-j} \frac{(M-j)!}{(M-j - k)!},
\end{align}
where~${P(T, k)}$ denotes the number of $k$-permutations of~$T$.
%
This gives an upper bound on the number of remaining prefix evaluations:
\begin{align}
\Remaining(\CurrentObj, \Queue)
\le \sum_{j=0}^J \Queue_j \left( \sum_{k=0}^{K-j} P(M-j, k) \right)
= \sum_{j=0}^J \Queue_j \left( \sum_{k=0}^{K-j} \frac{(M-j)!}{(M-j - k)!} \right).
\end{align}
\end{proof}
\end{arxiv}

\subsection{Lower bounds on antecedent support}
\label{sec:lb-support}

In this section, we give two lower bounds on the normalized support
of each antecedent in any optimal rule list;
both are related to the regularization parameter~$\Reg$.

\begin{theorem}[Lower bound on antecedent support]
\label{thm:min-capture}
Let ${\OptimalRL = (\Prefix, \Labels, \Default, K) \in \argmin_\RL \Obj(\RL, \x, \y)}$
be any optimal rule list, with objective~$\OptimalObj$.
%
For each antecedent~$p_k$ in prefix ${\Prefix = (p_1, \dots, p_K)}$,
\begin{arxiv}
the regularization parameter~$\Reg$ provides a lower bound
on the normalized support of~$p_k$,
\begin{align}
\Reg < \Supp(p_k, \x \given \Prefix).
\label{eq:min-capture}
\end{align}
\end{arxiv}
\begin{kdd}
the regularization parameter provides a lower bound,
${\Reg < \Supp(p_k, \x \given \Prefix)}$, on the normalized support of~$p_k$.
\end{kdd}
\end{theorem}

\begin{arxiv}
\begin{proof}
Let ${\OptimalRL = (\Prefix, \Labels, \Default, K)}$ be an optimal
rule list with prefix ${\Prefix = (p_1, \dots, p_K)}$
and labels ${\Labels = (q_1, \dots, q_K)}$.
%
Consider the rule list ${\RL = (\Prefix', \Labels', \Default', K-1)}$
derived from~$\OptimalRL$ by deleting a rule ${p_i \rightarrow q_i}$,
therefore ${\Prefix' = (p_1, \dots, p_{i-1}, p_{i+1}, \dots, p_K)}$
and ${\Labels' = (q_1, \dots, q_{i-1}, q'_{i+1}, \dots, q'_K)}$,
where~$q'_k$ need not be the same as~$q_k$, for ${k > i}$ and~${k = 0}$.

The largest possible discrepancy between~$\OptimalRL$ and~$\RL$ would occur
if~$\OptimalRL$ correctly classified all the data captured by~$p_i$,
while~$\RL$ misclassified these data.
%
This gives an upper bound:
\begin{align}
\Obj(\RL, \x, \y) = \Loss(\RL, \x, \y) + \Reg (K - 1)
&\le \Loss(\OptimalRL, \x, \y) + \Supp(p_i, \x \given \Prefix) + \Reg(K - 1) \nn \\
&= \Obj(\OptimalRL, \x, \y) + \Supp(p_i, \x \given \Prefix) - \Reg \nn \\
&= \OptimalObj + \Supp(p_i, \x \given \Prefix) - \Reg
\label{eq:ub-i}
\end{align}
where~$\Supp(p_i, \x \given \Prefix)$ is the normalized support of~$p_i$
in the context of~$\Prefix$, defined in~\eqref{eq:support-context},
and the regularization `bonus' comes from the fact that~$\RL$
is one rule shorter than~$\OptimalRL$.

At the same time, we must have ${\OptimalObj < \Obj(\RL, \x, \y)}$ for~$\OptimalRL$ to be optimal.
%
Combining this with~\eqref{eq:ub-i} and rearranging gives~\eqref{eq:min-capture},
therefore the regularization parameter~$\Reg$ provides a lower bound
on the support of an antecedent~$p_i$ in an optimal rule list~$\OptimalRL$.
\end{proof}
\end{arxiv}

Thus, we can prune a prefix~$\Prefix$ if any of its antecedents do not capture
more than a fraction~$\Reg$ of data, even if~${b(\Prefix, \x, \y) < \OptimalObj}$.
%
\begin{arxiv}
Notice that the
\end{arxiv}
\begin{kdd}
The
\end{kdd}
bound in Theorem~\ref{thm:min-capture}
depends on the antecedents, but not the label predictions,
and thus doesn't account for misclassification error.
%
\begin{arxiv}
Theorem~\ref{thm:min-capture-correct} gives a tighter bound
by leveraging this additional information, which specifically
tightens the upper bound on~$\Obj(\RL, \x, \y)$ in~\eqref{eq:ub-i}.
\end{arxiv}
\begin{kdd}
Theorem~\ref{thm:min-capture-correct} gives a tighter bound
by leveraging this information.
\end{kdd}

\begin{theorem}[Lower bound on accurate antecedent support]
\label{thm:min-capture-correct}
Let ${\OptimalRL \in \argmin_\RL \Obj(\RL, \x, \y)}$
be any optimal rule list, with objective~$\OptimalObj$;
let ${\OptimalRL = (\Prefix, \Labels, \Default, K)}$,
with prefix ${\Prefix = (p_1, \dots, p_K)}$
and labels ${\Labels = (q_1, \dots, q_K)}$.
%
For each rule~${p_k \rightarrow q_k}$ in~$\OptimalRL$,
define~$a_k$ to be the fraction of data that are captured by~$p_k$
and correctly classified:
\begin{align}
a_k \equiv \frac{1}{N} \sum_{n=1}^N
  \Cap(x_n, p_k \given \Prefix) \wedge \one [ q_k = y_n ].
\label{eq:rule-correct}
\end{align}
\begin{arxiv}
The regularization parameter~$\Reg$ provides a lower bound on~$a_k$:
\begin{align}
\Reg < a_k.
\label{eq:min-capture-correct}
\end{align}
\end{arxiv}
\begin{kdd}
The regularization parameter provides a lower bound, $\Reg < a_k$.
\end{kdd}
\end{theorem}

\begin{arxiv}
\begin{proof}
As in Theorem~\ref{thm:min-capture},
let ${\RL =  (\Prefix', \Labels', \Default', K-1)}$ be the rule list
derived from~$\OptimalRL$ by deleting a rule~${p_i \rightarrow q_i}$.
%
Now, let us define~$\Loss_i$ to be the portion of~$\OptimalObj$
due to this rule's misclassification error,
\begin{align}
\Loss_i \equiv \frac{1}{N} \sum_{n=1}^N
  \Cap(x_n, p_i \given \Prefix) \wedge \one [ q_i \neq y_n ].
\end{align}
The largest discrepancy between~$\OptimalRL$ and~$\RL$ would
occur if~$\RL$ misclassified all the data captured by~$p_i$.
%
This gives an upper bound on the difference between
the misclassification error of~$\RL$ and~$\OptimalRL$:
\begin{align}
\Loss(\RL, \x, \y) - \Loss(\OptimalRL, \x, \y)
&\le \Supp(p_i, \x \given \Prefix) - \Loss_i \nn \\
&= \frac{1}{N} \sum_{n=1}^N \Cap(x_n, p_i \given \Prefix)
  - \frac{1}{N} \sum_{n=1}^N
  \Cap(x_n, p_i \given \Prefix) \wedge \one [ q_i \neq y_n ] \nn \\
&= \frac{1}{N} \sum_{n=1}^N
  \Cap(x_n, p_i \given \Prefix) \wedge \one [ q_i = y_n ] = a_i,
\end{align}
where we defined~$a_i$ in~\eqref{eq:rule-correct}.
%
Relating this bound to the objectives of~$\RL$ and~$\OptimalRL$ gives
\begin{align}
\Obj(\RL, \x, \y) = \Loss(\RL, \x, \y) + \Reg (K - 1)
&\le \Loss(\OptimalRL, \x, \y) + a_i + \Reg(K - 1) \nn \\
&= \Obj(\OptimalRL, \x, \y) + a_i - \Reg \nn \\
&= \OptimalObj + a_i - \Reg
\label{eq:ub-ii}
\end{align}
Combining~\eqref{eq:ub-ii} with the requirement
${\OptimalObj < \Obj(\RL, \x, \y)}$ gives the bound~${\Reg < a_i}$.
\end{proof}
\end{arxiv}

Thus, we can prune a prefix if any of its rules do not capture
and correctly classify at least a fraction~$\Reg$ of data.
%
While the lower bound in Theorem~\ref{thm:min-capture} is a sub-condition
of the lower bound in Theorem~\ref{thm:min-capture-correct},
we can still leverage both -- since the sub-condition is easier to check,
checking it first can accelerate pruning.
%
In addition to applying Theorem~\ref{thm:min-capture} in the context of
constructing rule lists, we can furthermore apply it in the context of
rule mining~(\S\ref{sec:rule-mining}).
%
Specifically, it implies that we should only mine rules with
normalized support greater than~$\Reg$;
we need not mine rules with a smaller fraction of observations.
%
In contrast, we can only apply Theorem~\ref{thm:min-capture-correct}
in the context of constructing rule lists;
it depends on the misclassification error associated with each
rule in a rule list, thus it provides a lower bound on the number of
observations that each such rule must correctly classify.

\begin{arxiv}
\subsubsection{Some antecedent relationships that imply insufficient antecedent support}

\dots

\begin{comment}
Let us say that rule~$A$ dominates rule~$B$ if the data captured by~$B$ is a subset of the data captured by~$A$.
%
Rule~$B$ should never follow rule~$A$ in a rule list because it will never capture additional data;
this scenario is a special case where~\eqref{eq:min-capture} immediately applies.
%
More precisely, if~$\RL$ is a rule list that contains rule~$A$ and doesn't contain rule~$B$,
and~$\RL'$ is derived from~$\RL$ by inserting rule~$B$ anywhere after~$A$, then
\begin{align}
\Obj(\RL', \x, \y) = \Obj(\RL, \x, \y) + c \ge \Obj(\RL, \x, \y).
\end{align}
By considering rule semantics, it is easy to think of common situations
leading to dominance relationships between rules.
%
For example, if rule~$A$'s antecedent is~${(x_1 = 0)}$ and rule~$B$'s
antecedent is~${(x_1 = 0) \wedge (x_2 = 1)}$, then rule~$A$ dominates rule~$B$.
%
%Similar to symmetry-aware pruning, we achieve this by restricting how we grow a prefix, as informed by a hash map rdict that maps a rule R_i to a set of rules T_i = {R_k} such that R_i dominates every R_k in T_i. When growing a prefix that ends with rule R_i, we only append a rule R_k if it is not in T_i. Notice that the intersection of S_i and T_i is the empty set, thus the mappings represented by cdict and rdict can easily be combined into a single mapping.
\end{comment}
\end{arxiv}

\begin{arxiv}
\subsection{Upper bound on antecedent support}
\label{sec:ub-support}

In the previous section~(\S\ref{sec:lb-support}), we proved lower bounds
on antecedent support; in this section, we give an upper bound on
antecedent support.
%
Specifically, Theorem~\ref{thm:ub-support} shows that an antecedent's
support in a rule list cannot be too similar to the set of data not
captured by preceding antecedents in the rule list.

\begin{theorem}[Upper bound on antecedent support]
\label{thm:ub-support}
Let ${\RL^* = (\Prefix, \Labels, \Default, K) \in \argmin_\RL \Obj(\RL, \x, \y)}$
be any optimal rule list, and let
${\Prefix = (p_1, \dots, p_{j-1}, p_j, \dots, p_{K-1}, p_K)}$ be its prefix.
%
The last antecedent~$p_K$ in~$\Prefix$ has support
\begin{align}
\Supp(p_K, \x \given \Prefix) \le 1 - \Supp(\Prefix^{K-1}, \x) - \Reg,
\label{eq:ub-support-last}
\end{align}
where ${\Prefix^{K-1} = (p_1, \dots, p_{K-1})}$,
with equality implying that there also exists a shorter optimal rule list
${\RL' = (\Prefix^{K-1}, \Labels', \Default', K - 1) \in}$ ${\argmin_\RL \Obj(\RL, \x, \y)}$
with prefix~$\Prefix^{K-1}$.
%
For all ${k \le K - 1}$, every antecedent~$p_k$ in~$\Prefix$ has support
less than the fraction of all data not captured by preceding antecedents,
by an amount greater than the regularization parameter~$\Reg$:
\begin{align}
\Supp(p_k, \x \given \Prefix) < 1 - \Supp(\Prefix^{k-1}, \x) - \Reg,
\label{eq:ub-support}
\end{align}
where ${\Prefix^{k-1} = (p_1, \dots, p_{k-1})}$.
\end{theorem}

\begin{proof}
We begin by focusing on the last antecedent in a rule list.
%
Let ${\RL = (\Prefix, \Labels, \Default, K)}$
be a rule list with prefix ${\Prefix = (p_1, \dots, p_K)}$
and objective ${\Obj(\RL, \x, \y) \le \OptimalObj}$, where
${\OptimalObj \equiv \min_{\RLB} \Obj(\RLB, \x, \y)}$
is the optimal objective.
%
Also let ${\RL' = (\Prefix', \Labels', \Default', K + 1)}$
be a rule list whose prefix ${\Prefix' = (p_1, \dots, p_K, p_{K+1})}$
starts with~$\Prefix$ and ends with a new antecedent~$p_{K+1}$.
%
Suppose~$p_{K+1}$ in the context of~$\Prefix'$ captures nearly all
data not captured by~$\Prefix$, except for a fraction~$\epsilon$
upper bounded by the regularization parameter~$\Reg$:
\begin{align}
1 - \Supp(\Prefix, \x) - \Supp(p_{K+1}, \x \given \Prefix') \equiv \epsilon \le \Reg.
\end{align}
%
Since~$\Prefix'$ starts with~$\Prefix$,
its prefix misclassification error is at least as great;
the only discrepancy between the misclassification errors
of~$\RL$ and~$\RL'$ can come from the difference between the support of
the set of data not captured by~$\Prefix$ and the support of~$p_{K+1}$:
\begin{align}
| \Loss(\RL', \x, \y) - \Loss(\RL, \x, \y) | \le
1 - \Supp(\Prefix, \x) - \Supp(p_{K+1}, \x \given \Prefix') = \epsilon.
\end{align}
The best outcome for~$\RL'$ would occur if its misclassification
error were smaller than that of~$\RL$ by~$\epsilon$,
%\eg this could happen if all data not captured by~$\Prefix'$
%had the same class label, in which case the default rule of~$\RL'$
%would incur zero misclassification error.
% --> it's more complicated, would need to discuss minority class label, etc.
%
therefore
\begin{align}
\Obj(\RL', \x, \y) &= \Loss(\RL', \x, \y) + \Reg (K+1) \nn \\
&\ge \Loss(\RL, \x, \y) - \epsilon + \Reg(K+1)
= \Obj(\RL, \x, \y) - \epsilon + \Reg \ge \Obj(\RL, \x, \y) \ge \OptimalObj.
\end{align}
$\RL'$ is an optimal rule list,
\ie ${\RL' \in \argmin_{\RLB} \Obj(\RLB, \x, y)}$,
if and only if ${\Obj(\RL', \x, \y) = \Obj(\RL, \x, \y) = \OptimalObj}$,
which requires ${\epsilon = \Reg}$.
%
Otherwise, ${\epsilon < \Reg}$, in which case
\begin{align}
\Obj(\RL', \x, \y) \ge \Obj(\RL, \x, \y) - \epsilon + \Reg
> \Obj(\RL, \x, \y) \ge \OptimalObj,
\end{align}
\ie $\RL'$ is not optimal.
%
This proves the first half of Theorem~\ref{thm:ub-support}.

To finish, we prove the bound in~\eqref{eq:ub-support} by contradiction.
%
First, note that the data not captured by~$\Prefix'$
has normalized support~${\epsilon \le \Reg}$, \ie
\begin{align}
1 - \Supp(\Prefix', \x) = 1 - \Supp(\Prefix, \x) - \Supp(p_{K+1}, \x \given \Prefix') = \epsilon \le \Reg.
\end{align}
Thus for any rule list~$\RL''$ whose prefix
$\Prefix'' = (p_1, \dots, p_{K+1}, \dots, p_{K'})$ starts
with~$\Prefix'$ and ends with one or more additional rules,
each additional rule~$p_k$ has support
${\Supp(p_k, \x \given \Prefix'') \le \epsilon \le \Reg}$,
for all~${k > K+1}$.
%
By Theorem~\ref{thm:min-capture},
all of the additional rules have insufficient support,
therefore~$\Prefix''$ cannot be optimal,
\ie ${\RL'' \notin \argmin_{\RLB} \Obj(\RLB, \x, \y)}$.
\end{proof}

Similar to Theorem~\ref{thm:min-capture}, our lower bound on
antecedent support, we can apply Theorem~\ref{thm:ub-support}
in the contexts of both constructing rule lists and
rule mining~(\S\ref{sec:rule-mining}).
%
Theorem~\ref{thm:ub-support} implies that if we only seek a single
optimal rule list, then during branch-and-bound execution,
we can prune a prefix if we ever add an antecedent with support
too similar to the support of set of data not captured by the
preceding antecedents.
%
One way to view this result is that if
${\RL = (\Prefix, \Labels, \Default, K)}$
and ${\RL' = (\Prefix', \Labels', \Default', K + 1)}$
are rule lists such that~$\Prefix'$ starts with~$\Prefix$
and ends with an antecedent that captures all or nearly all
data not captured by~$\Prefix$, then the new rule in~$\RL'$
behaves similar to the default rule of~$\RL$.
%
As a result, the misclassification error of~$\RL'$ must be
similar to that of~$\RL$, and any reduction may not be
sufficient to offset the penalty for longer prefixes.
%
Furthermore, Theorem~\ref{thm:ub-support} implies that we should
only mine rules with normalized support less than ${1 - \Reg}$;
we need not mine rules with a larger fraction of observations.
\end{arxiv}

\begin{arxiv}
\subsection{Antecedent rejection and its propagation}
\label{sec:reject}

In this section, we demonstrate further consequences of
our lower~(\S\ref{sec:lb-support}) and upper
bounds~(\S\ref{sec:ub-support}) on antecedent support,
under a unified framework we refer to as antecedent rejection.
%
Let ${\Prefix = (p_1, \dots, p_K)}$ be a prefix,
and let~$p_k$ be an antecedent in~$\Prefix$.
%
Define~$p_k$ to have insufficient support in~$\Prefix$
if does not obey the bound in~\eqref{eq:min-capture}
of Theorem~\ref{thm:min-capture}.
%
Define~$p_k$ to have insufficient accurate support in~$\Prefix$
if it does not obey the bound in~\eqref{eq:min-capture-correct}
of Theorem~\ref{thm:min-capture-correct}.
%
Define~$p_k$ to have excessive support in~$\Prefix$ if it
does not obey the appropriate bound in Theorem~\ref{thm:ub-support},
\ie either it's the last antecedent and doesn't obey~\eqref{eq:ub-support},
or it's any other antecedent and doesn't obey~\eqref{eq:ub-support-last}.
%
If~$p_k$ in the context of~$\Prefix$ has insufficient support,
insufficient accurate support, or excessive support,
%or support too similar to the set of data not captured by
%preceding antecedents (Theorem~\ref{thm:ub-support}),
let us say that prefix~$\Prefix$ rejects antecedent~$p_K$.
%
Next, in Theorem~\ref{thm:reject}, we describe large classes of
related rule lists whose prefixes all reject the same antecedent.

\begin{theorem}[Antecedent rejection propagates]
\label{thm:reject}
For any prefix ${\Prefix = (p_1, \dots, p_K)}$,
let~$\StartContains(\Prefix)$ denote the set of all
prefixes~$\Prefix'$ such that
the set of all antecedents in~$\Prefix$ is a subset of
the set of all antecedents in~$\Prefix'$, \ie
\begin{align}
\StartContains(\Prefix) =
\{\Prefix' = (p'_1, \dots, p'_{K'})
~s.t.~ \{p_k : p_k \in \Prefix \} \subseteq
\{p'_\kappa : p'_\kappa \in \Prefix'\}, K' \ge K \}.
\label{eq:start-contains}
\end{align}
%
Let ${\RL = (\Prefix, \Labels, \Default, K)}$ be a rule list
with prefix ${\Prefix = (p_1, \dots, p_{K-1}, p_{K})}$,
such that~$\Prefix$ rejects its last antecedent~$p_{K}$,
either because~$p_{K}$ in the context of~$\Prefix$ has
insufficient support, insufficient accurate support,
or excessive support.
%
Let ${\Prefix^{K-1} = (p_1, \dots, p_{K-1})}$ be the
first~${K - 1}$ antecedents of~$\Prefix$.
%
Let ${\RLB = (\PrefixB, \LabelsB, \DefaultB, \kappa)}$
be any rule list with prefix
${\PrefixB = (P_1, \dots, P_{K'-1}, P_{K'}, \dots, P_{\kappa})}$
such that~$\PrefixB$ starts with ${\PrefixB^{K'-1} =}$
${(P_1, \dots, P_{K'-1}) \in}$ ${\StartContains(\Prefix^{K-1})}$
and antecedent~${P_{K'} = p_{K}}$.
%
It follows that prefix~$\PrefixB$ rejects~$P_{K'}$
for the same reason that~$\Prefix$ rejects~$p_{K}$,
and furthermore, $\RLB$~cannot be optimal, \ie
${\RLB \notin \argmin_{\RL^\dagger} \Obj(\RL^\dagger, \x, \y)}$.
\end{theorem}

\begin{proof}
Combine Propositions~\ref{prop:min-capture},
\ref{prop:min-capture-correct}, and~\ref{prop:ub-support}.
\end{proof}

\begin{proposition}[Insufficient antecedent support propagates]
\label{prop:min-capture}
Define~$\StartContains(\Prefix)$ as in~\eqref{eq:start-contains},
and let ${\Prefix = (p_1, \dots, p_{K-1}, p_{K})}$ be a prefix,
such that its last antecedent~$p_{K}$ has insufficient support,
\ie the opposite of the bound in~\eqref{eq:min-capture}:
${\Supp(p_K, \x \given \Prefix) \le \Reg}$.
%
Let ${\Prefix^{K-1} = (p_1, \dots, p_{K-1})}$,
and let ${\RLB = (\PrefixB, \LabelsB, \DefaultB, \kappa)}$
be any rule list with prefix
${\PrefixB = (P_1, \dots, P_{K'-1}, P_{K'}, \dots, P_{\kappa})}$,
such that~$\PrefixB$ starts with ${\PrefixB^{K'-1} =}$
${(P_1, \dots, P_{K'-1}) \in \StartContains(\Prefix^{K-1})}$
and~${P_{K'} = p_{K}}$.
%
It follows that~$P_{K'}$ has insufficient support in
prefix~$\PrefixB$, and furthermore, $\RLB$~cannot be optimal,
\ie ${\RLB \notin \argmin_{\RL} \Obj(\RL, \x, \y)}$.
\end{proposition}

\begin{proof}
The support of~$p_K$ in~$\Prefix$ depends only on the
set of antecedents in ${\Prefix^{K} = (p_1, \dots, p_{K})}$:
\begin{align}
\Supp(p_K, \x \given \Prefix)
= \frac{1}{N} \sum_{n=1}^N \Cap(x_n, p_K \given \Prefix)
&= \frac{1}{N} \sum_{n=1}^N \left( \neg\, \Cap(x_n, \Prefix^{K-1}) \right)
  \wedge \Cap(x_n, p_K) \nn \\
&= \frac{1}{N} \sum_{n=1}^N \left( \bigwedge_{k=1}^{K-1} \neg\, \Cap(x_n, p_k) \right)
  \wedge \Cap(x_n, p_K)
\le \Reg.
\end{align}
Similarly, the support of~$P_{K'}$ in~$\PrefixB$ depends only on
the set of antecedents in ${\PrefixB^{K'} = (P_1, \dots, P_{K'})}$:
\begin{align}
\Supp(P_{K'}, \x \given \PrefixB)
= \frac{1}{N} \sum_{n=1}^N \Cap(x_n, P_{K'} \given \PrefixB)
&= \frac{1}{N} \sum_{n=1}^N \left( \neg\, \Cap(x_n, \PrefixB^{K'-1}) \right)
  \wedge \Cap(x_n, P_{K'}) \nn \\
&= \frac{1}{N} \sum_{n=1}^N \left( \bigwedge_{k=1}^{K'-1} \neg\, \Cap(x_n, P_k) \right)
   \wedge \Cap(x_n, P_{K'}) \nn \\
&\le \frac{1}{N} \sum_{n=1}^N \left( \bigwedge_{k=1}^{K-1} \neg\, \Cap(x_n, p_k) \right)
  \wedge \Cap(x_n, P_{K'}) \nn \\
&= \frac{1}{N} \sum_{n=1}^N \left( \bigwedge_{k=1}^{K-1} \neg\, \Cap(x_n, p_k) \right)
  \wedge \Cap(x_n, p_{K}) \nn \\
&= \Supp(p_K, \x \given \Prefix) \le \Reg.
\label{ineq:supp}
\end{align}
The first inequality reflects the condition that
${\PrefixB^{K'-1} \in \StartContains(\Prefix^{K-1})}$,
which implies that the set of antecedents in~$\PrefixB^{K'-1}$
contains the set of antecedents in~$\Prefix^{K-1}$,
and the next equality reflects the fact that~${P_{K'} = p_K}$.
%
Thus,~$P_K'$ has insufficient support in prefix~$\PrefixB$,
therefore by Theorem~\ref{thm:min-capture}, $\RLB$~cannot be optimal,
\ie ${\RLB \notin \argmin_{\RL} \Obj(\RL, \x, \y)}$.
\end{proof}

\begin{proposition}[Insufficient accurate antecedent support propagates]
\label{prop:min-capture-correct}
Define~$\StartContains(\Prefix)$ as in~\eqref{eq:start-contains},
and let ${\RL = (\Prefix, \Labels, \Default, K)}$ be a rule list
with prefix ${\Prefix = (p_1, \dots, p_{K})}$
and labels ${\Labels = (q_1, \dots, q_{K})}$, such that
the last antecedent~$p_{K}$ has insufficient accurate support,
\ie the opposite of the bound in~\eqref{eq:min-capture-correct}:
\begin{align}
\frac{1}{N} \sum_{n=1}^N \Cap(x_n, p_K \given \Prefix) \wedge \one [ q_K = y_n ]
\le \Reg.
\end{align}
%
Let ${\Prefix^{K-1} = (p_1, \dots, p_{K-1})}$
and let ${\RLB = (\PrefixB, \LabelsB, \DefaultB, \kappa)}$
be any rule list with prefix ${\PrefixB = (P_1, \dots, P_{\kappa})}$
and labels ${\LabelsB = (Q_1, \dots, Q_{\kappa})}$,
such that~$\PrefixB$ starts with ${\PrefixB^{K'-1} =}$
${(P_1, \dots, P_{K'-1})}$ ${\in \StartContains(\Prefix^{K-1})}$
and ${P_{K'} = p_{K}}$.
%
It follows that~$P_{K'}$ has insufficient accurate support in
prefix~$\PrefixB$, and furthermore,
${\RLB \notin \argmin_{\RL^\dagger} \Obj(\RL^\dagger, \x, \y)}$.
\end{proposition}

\begin{proof}
The accurate support of~$P_{K'}$ in~$\PrefixB$ is insufficient:
\begin{align}
\frac{1}{N} \sum_{n=1}^N \Cap(x_n, P_{K'} \given \PrefixB) \wedge \one [ Q_{K'} = y_n ]
&= \frac{1}{N} \sum_{n=1}^N \left( \bigwedge_{k=1}^{K'-1} \neg\, \Cap(x_n, P_k) \right)
   \wedge \Cap(x_n, P_{K'}) \wedge \one [ Q_{K'} = y_n ] \nn \\
&\le \frac{1}{N} \sum_{n=1}^N \left( \bigwedge_{k=1}^{K-1} \neg\, \Cap(x_n, p_k) \right)
   \wedge \Cap(x_n, P_{K'}) \wedge \one [ Q_{K'} = y_n ] \nn \\
&= \frac{1}{N} \sum_{n=1}^N \left( \bigwedge_{k=1}^{K-1} \neg\, \Cap(x_n, p_k) \right)
   \wedge \Cap(x_n, p_K) \wedge \one [ Q_{K'} = y_n ] \nn \\
&= \frac{1}{N} \sum_{n=1}^N \Cap(x_n, p_K \given \Prefix) \wedge \one [ Q_{K'} = y_n ] \nn \\
&\le \frac{1}{N} \sum_{n=1}^N \Cap(x_n, p_K \given \Prefix) \wedge \one [ q_{K} = y_n ]
\le \Reg.
\end{align}
The first inequality reflects the condition that
${\PrefixB^{K'-1} \in \StartContains(\Prefix^{K-1})}$,
the next equality reflects the fact that~${P_{K'} = p_K}$.
%
For the following equality, notice that~$Q_{K'}$ is the majority
class label of data captured by~$P_{K'}$ in~$\PrefixB$, and~$q_K$
is the majority class label of data captured by~$P_K$ in~$\Prefix$,
and recall from~\eqref{ineq:supp} that
${\Supp(P_{K'}, \x \given \PrefixB) \le \Supp(p_{K}, \x \given \Prefix)}$.
%
By Theorem~\ref{thm:min-capture-correct},
${\RLB \notin \argmin_{\RL^\dagger} \Obj(\RL^\dagger, \x, \y)}$.
\end{proof}

\begin{proposition}[Excessive antecedent support propagates]
\label{prop:ub-support}
Define~$\StartContains(\Prefix)$ as in~\eqref{eq:start-contains},
and let ${\Prefix = (p_1, \dots, p_{K})}$ be a prefix,
such that its last antecedent~$p_{K}$ has excessive support,
\ie the opposite of the bound in~\eqref{eq:ub-support-last}:
\begin{align}
\Supp(p_K, \x \given \Prefix) \ge 1 - \Supp(\Prefix^{K-1}, \x) - \Reg.
\end{align}
Let ${\Prefix^{K-1} = (p_1, \dots, p_{K-1})}$;
let ${\RLB = (\PrefixB, \LabelsB, \DefaultB, \kappa)}$
be any rule list with prefix
${\PrefixB = (P_1, \dots, P_{\kappa})}$
such that~$\PrefixB$ starts with ${\PrefixB^{K'-1} =}$
${(P_1, \dots, P_{K'-1}) \in \StartContains(\Prefix^{K-1})}$
and~${P_{K'} = p_{K}}$.
%
It follows that~$P_{K'}$ has excessive support in prefix~$\PrefixB$,
and furthermore, ${\RLB \notin \argmin_{\RL} \Obj(\RL, \x, \y)}$.
\end{proposition}

\begin{proof}
Since ${\PrefixB^{K'} = (P_1, \dots, P_{K'})}$
contains all the antecedents in~$\Prefix$, we have that
\begin{align}
\Supp(\PrefixB^{K'}, \x) \ge \Supp(\Prefix, \x).
\end{align}
Expanding these two terms gives
\begin{align}
\Supp(\PrefixB^{K'}, \x)
&= \Supp(\PrefixB^{K'-1}, \x) + \Supp(P_{K'}, \x \given \PrefixB) \nn \\
&\ge \Supp(\Prefix, \x)
= \Supp(\Prefix^{K-1}, \x) + \Supp(p_K, \x \given \Prefix)
\ge 1 - \Reg.
\end{align}
Rearranging gives
\begin{align}
\Supp(P_{K'}, \x \given \PrefixB)
\ge 1 - \Supp(\PrefixB^{K'-1}, \x) - \Reg,
\end{align}
thus~$P_{K'}$ has excessive support in~$\PrefixB$.
%
By Theorem~\ref{thm:ub-support},
${\RLB \notin \argmin_{\RL} \Obj(\RL, \x, \y)}$.
\end{proof}

Theorem~\ref{thm:reject} implies potentially significant
computational savings.
%
During branch-and-bound execution, if we ever encounter a
prefix ${\Prefix = (p_1, \dots, p_{K-1}, p_K)}$ that rejects its
last antecedent~$p_K$, then we can prune~$\Prefix$.
%
Furthermore, we can also prune \emph{any} prefix~$\Prefix'$
whose antecedents contains the set of antecedents in~$\Prefix$,
in almost any order, with the constraint that all antecedent
in ${\{p_1, \dots, p_{K-1}\}}$ precede~$p_K$.

%Let Q be P's parent. If Q rejects a rule, then P will also reject that rule. This 'inheritance' of rejected rules only depends on which data are captured by Q, and doesn't actually depend on the order of rules in Q. Let S be the set of rules formed from (K-1) rules of P, in any order. P inherits rejected rules from any elements of S. Because of our symmetry-based garbage collection of prefixes equivalent up to a permutation, there are at most K elements of S in the cache; we can identify these via the inverse canonical map (ICM) that maps an ordered prefix to its permutation in the cache. We thus lazily initialize the list of P's reject list of rejected rules.

\end{arxiv}

\subsection{Equivalent support bound for symmetry-aware garbage collection}
\label{sec:equivalent}

Let~$\PrefixB$ be a prefix, and let~$\xi(\PrefixB)$ be the set
of all prefixes that capture exactly the same data as~$\PrefixB$.
%
Now, let~$\RL$ be a rule list with prefix~$\Prefix$
in~$\xi(\PrefixB)$, such that~$\RL$ has the minimum objective
over all rule lists with prefixes in~$\xi(\PrefixB)$.
%
Finally, let~$\RL'$ be a rule list whose prefix~$\Prefix'$
starts with~$\Prefix$, such that~$\RL'$ has the minimum objective
over all rule lists whose prefixes start with~$\Prefix$.
%
Theorem~\ref{thm:equivalent} below implies that~$\RL'$ also has
the minimum objective over all rule lists whose prefixes start with
\emph{any} prefix in~$\xi(\PrefixB)$.

\begin{theorem}[Equivalent support bound]
\label{thm:equivalent}
\begin{arxiv}
Define ${\StartsWith(\Prefix) = }$
${\{(\Prefix', \Labels', \Default', K') : \Prefix' \textnormal{ starts with } \Prefix \}}$
to be the set of all rule lists whose prefixes start with~$\Prefix$.
\end{arxiv}
\begin{kdd}
Define $\StartsWith(\Prefix)$
to be the set of all rule lists whose prefix starts with~$\Prefix$,
as in~\eqref{eq:starts-with}.
\end{kdd}
%
Let ${\RL = (\Prefix, \Labels, \Default, K)}$
be a rule list with prefix ${\Prefix = (p_1, \dots, p_K)}$,
and let ${\RLB = (\PrefixB, \LabelsB, \DefaultB, \kappa)}$
be a rule list with prefix ${\PrefixB = (P_1, \dots, P_{\kappa})}$,
such that~$\Prefix$ and~$\PrefixB$ capture the same data,~\ie
\begin{align}
\{x_n : \Cap(x_n, \Prefix)\} = \{x_n : \Cap(x_n, \PrefixB)\}.
\end{align}
%
If the objective lower bounds of~$\RL$ and~$\RLB$
obey ${b(\Prefix, \x, \y) \le b(\PrefixB, \x, \y)}$,
then the objective of the optimal rule list in~$\StartsWith(\Prefix)$ gives a
lower bound on the objective of the optimal rule list in~$\StartsWith(\PrefixB)$:
\begin{align}
\min_{\RL' \in \StartsWith(\Prefix)} \Obj(\RL', \x, \y)
\le \min_{\RLB' \in \StartsWith(\PrefixB)} \Obj(\RLB', \x, \y).
\label{eq:permutation}
\end{align}
\end{theorem}

\begin{arxiv}
\begin{proof}
We begin by defining four related rule lists.
%
First, let ${\RL = (\Prefix, \Labels, \Default, K)}$
be a rule list with prefix ${\Prefix = (p_1, \dots, p_K)}$
and labels ${\Labels = (q_1, \dots, q_K)}$.
%
Second, let ${\RLB = (\PrefixB, \LabelsB, \DefaultB, \kappa)}$
be a rule list with prefix ${\PrefixB = (P_1, \dots, P_\kappa)}$
that captures the same data as~$\Prefix$,
and labels ${\LabelsB = (Q_1, \dots, Q_\kappa)}$.
%
Third, let ${\RL' = (\Prefix', \Labels', \Default', K') \in}$
${\StartsWith(\Prefix)}$ be any rule list
whose prefix starts with~$\Prefix$, such that~${K' \ge K}$.
%
Denote the prefix and labels of~$\RL'$ by
${\Prefix' = (p_1, \dots, p_K, p_{K+1}, \dots, p_{K'})}$
and ${\Labels = (q_1, \dots, q_{K'})}$, respectively.
%
Finally, define
${\RLB' = (\PrefixB', \LabelsB', \DefaultB', \kappa') \in \StartsWith(\PrefixB)}$
to be the `analogous' rule list, \ie whose prefix
${\PrefixB' = (P_1, \dots, P_\kappa, P_{\kappa+1}, \dots, P_{\kappa'})
= (P_1, \dots, P_\kappa, p_{K+1}, \dots, p_{K'})}$
starts with~$\PrefixB$ and ends with the same ${K'-K}$
antecedents as~$\Prefix'$.
%
Let ${\LabelsB' = (Q_1, \dots, Q_{\kappa'})}$
denote the labels of~$\RLB'$.

Next, we claim that the difference in the objectives
of rule lists~$\RL'$ and~$\RL$ is the same as the difference
in the objectives of rule lists~$\RLB'$ and~$\RLB$.
%
Let us expand the first difference as
\begin{align}
\Obj(\RL',& \x, \y) - \Obj(\RL, \x, \y)
  = \Loss(\RL', \x, \y) + \Reg K' - \Loss(\RL, \x, \y) - \Reg K \nn \\
&= \Loss_p(\Prefix', \Labels', \x, \y) + \Loss_0(\Prefix', \Default', \x, \y)
  - \Loss_p(\Prefix, \Labels, \x, \y) - \Loss_0(\Prefix, \Default, \x, \y)
  + \Reg (K' - K).
\end{align}
Similarly, let us expand the second difference as
\begin{align}
\Obj(\RLB',& \x, \y) - \Obj(\RLB, \x, \y)
  = \Loss(\RLB', \x, \y) + \Reg \kappa' - \Loss(\RLB, \x, \y) - \Reg \kappa \nn \\
&= \Loss_p(\PrefixB', \LabelsB', \x, \y) + \Loss_0(\PrefixB', \DefaultB', \x, \y)
  - \Loss_p(\PrefixB, \LabelsB, \x, \y) - \Loss_0(\PrefixB, \DefaultB, \x, \y)
  + \Reg (K' - K),
\end{align}
where we have used the fact that ${\kappa' - \kappa = K' - K}$.

The prefixes~$\Prefix$ and~$\PrefixB$ capture the same data.
%
Equivalently, the set of data that is not captured by~$\Prefix$
is the same as the set of data that is not captured by~$\PrefixB$, \ie
\begin{align}
\{x_n : \neg\, \Cap(x_n, \Prefix)\} = \{x_n : \neg\, \Cap(x_n, \PrefixB)\}.
\end{align}
Thus, the corresponding rule lists~$\RL$ and~$\RLB$
share the same default rule, \ie ${\Default = \DefaultB}$,
yielding the same default rule misclassification error:
\begin{align}
\Loss_0(\Prefix, \Default, \x, \y) = \Loss_0(\PrefixB, \DefaultB, \x, \y).
\end{align}
Similarly, prefixes~$\Prefix'$ and~$\PrefixB'$ capture
the same data, and thus rule lists~$\RL'$ and~$\RLB'$
have the same default rule misclassification error:
\begin{align}
\Loss_0(\Prefix, \Default, \x, \y) = \Loss_0(\PrefixB, \DefaultB, \x, \y).
\end{align}

At this point, to demonstrate our claim relating the objectives
of~$\RL$, $\RL'$, $\RLB$, and~$\RLB'$, what remains is to
show that the difference in the misclassification errors
of prefixes~$\Prefix'$ and~$\Prefix$ is the same as that
between~$\PrefixB'$ and~$\PrefixB$.
%
We can expand the first difference as
\begin{align}
\Loss_p(\Prefix', \Labels', \x, \y) - \Loss_p(\Prefix, \Labels, \x, \y)
%&= \frac{1}{N} \sum_{n=1}^N \sum_{k=1}^{K'}
%  \one [ \Cap(x_n, p_k \given \Prefix') \wedge (q_k \neq y_n) ]
%  - \frac{1}{N} \sum_{n=1}^N \sum_{k=1}^K
%  \one [ \Cap(x_n, p_k \given \Prefix) \wedge (q_k \neq y_n) ] \\
&= \frac{1}{N} \sum_{n=1}^N \sum_{k=K+1}^{K'}
  \Cap(x_n, p_k \given \Prefix') \wedge \one [ q_k \neq y_n ],
\end{align}
where we have used the fact that since~$\Prefix'$
starts with~$\Prefix$, the first~$K$ rules in~$\Prefix'$
make the same mistakes as those in~$\Prefix$.
%
Similarly, we can expand the second difference as
\begin{align}
\Loss_p(\PrefixB', \LabelsB', \x, \y) - \Loss_p(\PrefixB, \LabelsB, \x, \y)
&= \frac{1}{N} \sum_{n=1}^N \sum_{k=\kappa+1}^{\kappa'}
  \Cap(x_n, P_k \given \PrefixB') \wedge \one [ Q_k \neq y_n ] \nn \\
&= \frac{1}{N} \sum_{n=1}^N \sum_{k=K+1}^{K'}
  \Cap(x_n, p_k \given \PrefixB') \wedge \one [ Q_k \neq y_n ] \nn \\
&= \frac{1}{N} \sum_{n=1}^N \sum_{k=K+1}^{K'}
  \Cap(x_n, p_k \given \Prefix') \wedge \one [ q_k \neq y_n ] \label{eq:third} \\
&= \Loss_p(\Prefix', \Labels', \x, \y) - \Loss_p(\Prefix, \Labels, \x, \y) \nn.
\end{align}
To justify the equality in~\eqref{eq:third}, we observe first that
prefixes~$\PrefixB'$ and~$\Prefix'$ start with~$\kappa$ and~$K$
antecedents, respectively, that capture the same data.
%
Second, prefixes~$\PrefixB'$ and~$\Prefix'$ end with exactly
the same ordered list of~${K' - K}$ antecedents,
therefore for any~${k = 1, \dots, K' - K}$,
antecedent ${P_{\kappa + k} = p_{K + k}}$ in~$\PrefixB'$
captures the same data as~$p_{K + k}$ captures in~$\Prefix'$.
%
It follows that the corresponding labels are all equivalent, \ie
${Q_{\kappa + k} = q_{K + k}}$, for all~${k = 1, \dots, K' - K}$,
and consequently, the prefix misclassification error associated
with the last~${K' - K}$ antecedents of~$\Prefix'$ is the same
as that of~$\PrefixB'$.
%
We have therefore shown that the difference between the objectives
of~$\RL'$ and~$\RL$ is the same as that between~$\RLB'$ and~$\RLB$, \ie
\begin{align}
\Obj(\RL', \x, \y) - \Obj(\RL, \x, \y)
= \Obj(\RLB', \x, \y) - \Obj(\RLB, \x, \y).
\label{eq:equiv-analogous}
\end{align}

Next, suppose that the objective lower bounds of~$\RL$ and~$\RLB$
obey ${b(\Prefix, \x, \y) \le b(\PrefixB, \x, \y)}$, thus
\begin{align}
\Obj(\RL, \x, \y)
&= \Loss_p(\Prefix, \Labels, \x, \y) + \Loss_0(\Prefix, \Default, \x, \y) + \Reg K \nn \\
&= b(\Prefix, \x, \y) + \Loss_0(\Prefix, \Default, \x, \y) \nn \\
&\le b(\PrefixB, \x, \y) + \Loss_0(\Prefix, \Default, \x, \y)
= b(\PrefixB, \x, \y) + \Loss_0(\PrefixB, \DefaultB, \x, \y)
= \Obj(\RLB, \x, \y).
\label{eqref:equiv-ineq}
\end{align}
Now let~$\RL^*$ be an optimal rule list with prefix
constrained to start with~$\Prefix$,
\begin{align}
\RL^* \in \argmin_{\RL^\dagger \in \StartsWith(\Prefix)} \Obj(\RL^\dagger, \x, \y),
\end{align}
and let~$K^*$ be the length of~$\RL^*$.
%
Let~$\RLB^*$ be the analogous $\kappa^*$-rule list whose prefix starts
with~$\PrefixB$ and ends with the same~${K^* - K}$ antecedents as~$\RL^*$,
where~${\kappa^* = \kappa + K^* - K}$.
%
By~\eqref{eq:equiv-analogous},
\begin{align}
\Obj(\RL^*, \x, \y) - \Obj(\RL, \x, \y)
= \Obj(\RLB^*, \x, \y) - \Obj(\RLB, \x, \y).
\label{eq:equiv-diff}
\end{align}
Furthermore, we claim that~$\RLB^*$ is an optimal rule list
with prefix constrained to start with~$\PrefixB$,
\begin{align}
\RLB^* \in \argmin_{\RLB^\dagger \in \StartsWith(\PrefixB)} \Obj(\RLB^\dagger, \x, \y).
\label{eq:equiv-analogous-optimal}
\end{align}

To demonstrate~\eqref{eq:equiv-analogous-optimal},
we consider two separate scenarios.
%
In the first scenario, prefixes~$\Prefix$ and~$\PrefixB$
are composed of the same antecedents,
\ie the two prefixes are equivalent up to a permutation of
their antecedents, and as a consequence,
${\kappa = K}$ and~${\kappa^* = K^*}$.
%
Here, every rule list~${\RL'' \in \StartsWith(\Prefix)}$
that starts with~$\Prefix$
has an analogue~${\RLB'' \in \StartsWith(\PrefixB)}$
that starts with~$\PrefixB$,
such that~$\RL''$ and~$\RLB''$ obey~\eqref{eq:equiv-analogous},
and vice versa, and thus~\eqref{eq:equiv-analogous-optimal}
is a direct consequence of~\eqref{eq:equiv-diff}.

In the second scenario, prefixes~$\Prefix$ and~$\PrefixB$
are not composed of the same antecedents.
%
Define~${\phi = \{p_k : (p_k \in \Prefix) \wedge (p_k \notin \PrefixB)\}}$
to be the set of antecedents in~$\Prefix$ that are not in~$\PrefixB$,
and define~${\Phi = \{P_k : (P_k \in \PrefixB) \wedge (P_k \notin \Prefix)\}}$
to be the set of antecedents in~$\PrefixB$ that are not in~$\Prefix$;
either~${\phi \neq \emptyset}$, or~${\Phi \neq \emptyset}$, or both.

Suppose~${\phi \neq \emptyset}$, and let~${p \in \phi}$
be an antecedent in~$\phi$.
%
It follows that there exists a subset of rule lists
in~$\StartsWith(\PrefixB)$ that do not have analogues
in~$\StartsWith(\Prefix)$.
%
Let~${\RLB'' \in \StartsWith(\PrefixB)}$ be such a rule list,
such that its prefix ${\PrefixB'' = (P_1, \dots, P_\kappa, \dots, p, \dots)}$
starts with~$\PrefixB$ and contains~$p$ among its remaining antecedents.
%
Since~$p$ captures a subset of the data that~$\Prefix$ captures,
and~$\PrefixB$ captures the same data as~$\Prefix$,
it follows that~$p$ doesn't capture any data in~$\PrefixB''$, \ie
\begin{align}
\frac{1}{N} \sum_{n=1}^N \Cap(x_n, p \given \PrefixB'') = 0 \le \Reg.
\end{align}
By Theorem~\ref{thm:min-capture}, antecedent~$p$ has insufficient
support in~$\RLB''$, and thus~$\RLB''$ cannot be optimal, \ie
${\RLB'' \notin}$ ${\argmin_{\RLB^\dagger \in \StartsWith(\PrefixB)} \Obj(\RLB^\dagger, \x, \y)}$.
%
By a similar argument, if~${\Phi \neq \emptyset}$
and~${P \in \Phi}$, and~${\RL'' \in \StartsWith(\Prefix)}$
is any rule list whose prefix starts with~$\Prefix$
and contains antecedent~$P$, then~$\RL''$ cannot be optimal, \ie
${\RL'' \notin \argmin_{\RL^\dagger \in \StartsWith(\Prefix)} \Obj(\RL^\dagger, \x, \y)}$.

To finish justifying claim~\eqref{eq:equiv-analogous-optimal}
for the second scenario, first define
\begin{align}
\tau(\Prefix, \Phi) \equiv
  \{\RL'' = (\Prefix'', \Labels'', \Default'', K'') :
    \RL'' \in \StartsWith(\Prefix) \textnormal{ and }
    p_k \notin \Phi, \forall p_k \in \Prefix''\} \subset \StartsWith(\Prefix)
\end{align}
to be the set of all rule lists whose prefixes start with~$\Prefix$
and don't contain any antecedents in~$\Phi$.
%
Now, recognize that the optimal prefixes in~$\tau(\Prefix, \Phi)$
and~$\StartsWith(\Prefix)$ are the same, \ie
\begin{align}
\argmin_{\RL^\dagger \in \tau(\Prefix, \Phi)} \Obj(\RL^\dagger, \x, \y)
= \argmin_{\RL^\dagger \in \StartsWith(\Prefix)} \Obj(\RL^\dagger, \x, \y),
\end{align}
and similarly, the optimal prefixes in~$\tau(\PrefixB, \phi)$
and~$\StartsWith(\PrefixB)$ are the same, \ie
\begin{align}
\argmin_{\RLB^\dagger \in \tau(\PrefixB, \phi)} \Obj(\RLB^\dagger, \x, \y)
= \argmin_{\RLB^\dagger \in \StartsWith(\PrefixB)} \Obj(\RLB^\dagger, \x, \y).
\end{align}
Since we have shown that every~${\RL'' \in \tau(\Prefix, \Phi)}$
has a direct analogue~${\RLB'' \in \tau(\PrefixB, \phi)}$,
such that~$\RL''$ and~$\RLB''$ obey~\eqref{eq:equiv-analogous},
and vice versa, we again have~\eqref{eq:equiv-analogous-optimal}
as a consequence of~\eqref{eq:equiv-diff}.

We can now finally combine~\eqref{eqref:equiv-ineq}
and~\eqref{eq:equiv-analogous-optimal} to obtain
\begin{align}
\min_{\RL' \in \StartsWith(\Prefix)} \Obj(\RL', \x, \y)
= \Obj(\RL^*, \x, \y) \le \Obj(\RLB^*, \x, \y)
= \min_{\RLB' \in \StartsWith(\PrefixB)} \Obj(\RLB', \x, \y).
\end{align}
\end{proof}
\end{arxiv}

Thus, if prefixes~$\Prefix$ and~$\PrefixB$ capture the same data,
and their objective lower bounds obey
${b(\Prefix, \x, \y) \le b(\PrefixB, \x, \y)}$,
Theorem~\ref{thm:equivalent} implies that we can prune~$\PrefixB$.
%
In our implementation, we call this symmetry-aware garbage collection.
%
Next, in Sections~\ref{sec:permutation} and~\ref{sec:permutation-counting},
we highlight and analyze the special case of prefixes that capture
the same data because they contain the same antecedents.

\subsubsection{Permutation bound for permutation-aware garbage collection}
\label{sec:permutation}

Let~${P = \{p_k\}_{k=1}^K}$ be a set of~$K$ antecedents,
and let~$\Pi$ be the set of all $K$-prefixes corresponding to
permutations of antecedents in~$P$.
%
Now, let~$\RL$ be a rule list with prefix~$\Prefix$ in~$\Pi$,
such that~$\RL$ has the minimum objective over all rule lists
with prefixes in~$\Pi$.
%
Finally, let~$\RL'$ be a rule list whose prefix~$\Prefix'$
starts with~$\Prefix$, such that~$\RL'$ has the minimum objective
over all rule lists whose prefixes start with~$\Prefix$.
%
Corollary~\ref{thm:permutation} below,
which can be viewed as special case of Theorem~\ref{thm:equivalent},
implies that~$\RL'$ also has the minimum objective over all
rule lists whose prefixes start with \emph{any} prefix in~$\Pi$.

\begin{corollary}[Permutation bound]
\label{thm:permutation}
Let~$\pi$ be any permutation of ${\{1, \dots, K\}}$,
\begin{arxiv}
and define ${\StartsWith(\Prefix) = }$
${\{(\Prefix', \Labels', \Default', K') : \Prefix' \textnormal{ starts with } \Prefix \}}$
to be the set of all rule lists whose prefixes start with~$\Prefix$.
\end{arxiv}
\begin{kdd}
and define $\StartsWith(\Prefix)$
to be the set of all rule lists whose prefix starts with~$\Prefix$,
as in~\eqref{eq:starts-with}.
\end{kdd}
%
Let ${\RL = (\Prefix, \Labels, \Default, K)}$
and ${\RLB = (\PrefixB, \LabelsB, \DefaultB, K)}$
denote rule lists with prefixes ${\Prefix = (p_1, \dots, p_K)}$
and ${\PrefixB = (p_{\pi(1)}, \dots, p_{\pi(K)})}$,
respectively, \ie the antecedents in~$\PrefixB$
correspond to a permutation of the antecedents in~$\Prefix$.
%
If the objective lower bounds of~$\RL$ and~$\RLB$
obey ${b(\Prefix, \x, \y) \le b(\PrefixB, \x, \y)}$,
then the objective of the optimal rule list in~$\StartsWith(\Prefix)$ gives a
lower bound on the objective of the optimal rule list in~$\StartsWith(\PrefixB)$:
\begin{align}
\min_{\RL' \in \StartsWith(\Prefix)} \Obj(\RL', \x, \y)
\le \min_{\RLB' \in \StartsWith(\PrefixB)} \Obj(\RLB', \x, \y).
\label{eq:permutation}
\end{align}
\end{corollary}

\begin{proof}
Since prefixes~$\Prefix$ and~$\PrefixB$ contain
the same antecedents, they both capture the same data.
Thus, we can apply Theorem~\ref{thm:equivalent}.
\end{proof}

Thus if prefixes~$\Prefix$ and~$\PrefixB$ have the same antecedents,
up to a permutation, and their objective lower bounds
obey~${b(\Prefix, \x, \y) \le}$ ${b(\PrefixB, \x, \y)}$,
Corollary~\ref{thm:permutation} implies that we can prune~$\PrefixB$.
%
We call this permutation-aware
garbage collection, and we illustrate the subsequent
computational savings next in~\S\ref{sec:permutation-counting}.

\subsubsection{Upper bound on prefix evaluations with permutation-aware garbage collection}
\label{sec:permutation-counting}

Here, we present an upper bound on the total number of prefix
evaluations that accounts for the effect of permutation-aware
garbage collection~(\S\ref{sec:permutation}).
%
Since every subset of~$K$ antecedents generates an equivalence
class of~$K!$ prefixes equivalent up to permutation, permutation-aware
garbage collection dramatically prunes the search space.

First, notice that Algorithm~\ref{alg:branch-and-bound} describes a
breadth-first exploration of the state space of rule lists.
%
Now suppose we integrate permutation-aware garbage collection into
our execution of branch-and-bound, so that after evaluating
prefixes of length~$K$, we only keep a single best prefix
from each set of prefixes equivalent up to a permutation.

\begin{theorem}[Upper bound on the total number of prefix evaluations with
permutation-aware garbage collection]
%
Consider a state space of all rule lists formed from a set~$\RuleSet$
of~$M$ antecedents, and consider the branch-and-bound algorithm with
permutation-aware garbage collection.
%
Define $\TotalRemaining(\RuleSet)$ to be the total number of prefixes evaluated.
%
For any set~$\RuleSet$ of $M$ rules,
\begin{align}
\TotalRemaining(\RuleSet)
\le  1 + \sum_{k=1}^K \frac{1}{(k - 1)!} \cdot \frac{M!}{(M - k)!},
\end{align}
where ${K = \min(\lfloor 1 / 2 \Reg \rfloor, M)}$.
\end{theorem}

\begin{proof}
By Corollary~\ref{cor:ub-prefix-length},
${K \equiv \min(\lfloor 1 / 2 \Reg \rfloor, M)}$
gives an upper bound on the length of any optimal rule list.
%
The algorithm begins by evaluating the empty prefix,
followed by~$M$ prefixes of length~${k=1}$,
then~${P(M, 2)}$ prefixes of length~${k=2}$,
where~${P(M, 2)}$ is the number of size-2 subsets of~$\{1, \dots, M \}$.
%
Before proceeding to length~${k=3}$, we keep only~${C(M, 2)}$
prefixes of length~${k=2}$, where~${C(M, k)}$ denotes the
number of $k$-combinations of~$M$.
%
Now, the number of length~${k=3}$ prefixes we evaluate is~${C(M, 2) (M - 2)}$.
%
Propagating this forward gives
\begin{arxiv}
\begin{align}
\TotalRemaining(\RuleSet) \le 1 + \sum_{k=1}^K C(M, k-1) (M - k + 1)
%= 1 + \sum_{k=1}^K {M \choose k-1}(M - k + 1)
%= 1 + \sum_{k=1}^K \frac{M! (M - k + 1)}{(k - 1)! (M - k + 1)!}
= 1 + \sum_{k=1}^K \frac{1}{(k - 1)!} \cdot \frac{M!}{(M - k)!}.
\end{align}
\end{arxiv}
\begin{align}
\TotalRemaining(\RuleSet) \le 1 + \sum_{k=1}^K C(M, k-1) (M - k + 1).
\end{align}
\end{proof}

Pruning based on permutation symmetries thus yields significant
computational savings.
%
Let us compare, for example, to the na\"ive number of prefix evaluations
given by the upper bound in Proposition~\ref{thm:ub-total-eval}.
%
If~${M = 100}$ and~${K = 5}$, then the na\"ive number is about
${9.1 \times 10^9}$, while the reduced number due to permutation-aware
garbage collection is about ${3.9 \times 10^8}$,
which is smaller by a factor of about~23.
%
If~${M=1000}$ and~${K = 10}$, the number of evaluations falls from
about~${9.6 \times 10^{29}}$ to about~${2.7 \times 10^{24}}$,
which is smaller by a factor of about~360,000.
%
% ELA : someone please double-check these numbers :)

While a number like~$10^{24}$ seems infeasibly enormous,
it does not represent the number of rule lists we evaluate.
%
As we show in ~(\S\ref{sec:experiments}),
our permutation bound in Corollary~\ref{thm:permutation}
and our other bounds together conspire to reduce the search space
to a size manageable on a single computer.
%
The choice of ${M=1000}$ and ${K=10}$ in our example above
corresponds to the state space size our efforts target.
%
${K=10}$ rules represents a (heuristic) upper limit on
the size of an interpretable rule list,
and ${M=1000}$ represents the approximate number of rules
with sufficiently high support (Theorem~\ref{thm:min-capture})
we expect to obtain via rule mining~(\S\ref{sec:setup}),
for many datasets that might be used for constructing interpretable models.

\begin{arxiv}
\subsection{Similar support bound}
\label{sec:similar}

We now present a relaxation of our equivalent support bound
from Theorem~\ref{thm:equivalent}.

\begin{theorem}[Similar support bound]
\label{thm:similar}
Define ${\StartsWith(\Prefix) = }$
${\{(\Prefix', \Labels', \Default', K') : \Prefix' \textnormal{ starts with } \Prefix \}}$
to be the set of all rule lists whose prefixes start with~$\Prefix$.
%
Let ${\Prefix = (p_1, \dots, p_K)}$ and
${\PrefixB = (P_1, \dots, P_{\kappa})}$ be prefixes
that capture nearly the same data.
%
Specifically, define~$\omega$ to be the normalized support
of data captured by~$\Prefix$ and not captured by~$\PrefixB$, \ie
%and let us require that~${\omega \le \Reg}$, \ie
\begin{align}
\omega \equiv \frac{1}{N} \sum_{n=1}^N
  \neg\, \Cap(x_n, \PrefixB)
  \wedge \Cap(x_n, \Prefix). % \le \Reg,
\label{eq:omega}
\end{align}
%where~$\Reg$ is the regularization parameter.
%
Similarly, define~$\Omega$ to be the normalized support
of data captured by~$\PrefixB$ and not captured by~$\Prefix$, \ie
%and let us require that~${\Omega \le \Reg}$, \ie
\begin{align}
\Omega \equiv \frac{1}{N} \sum_{n=1}^N
  \neg\, \Cap(x_n, \Prefix)
  \wedge \Cap(x_n, \PrefixB). %\le \Reg.
\label{eq:big-omega}
\end{align}
We can bound the difference between the objectives of the
optimal rule lists in~$\StartsWith(\Prefix)$
and~$\StartsWith(\PrefixB)$ as follows:
\begin{align}
\min_{\RLB^\dagger \in \StartsWith(\PrefixB)} \Obj(\RLB^\dagger, \x, \y)
- \min_{\RL^\dagger \in \StartsWith(\Prefix)} \Obj(\RL^\dagger, \x, \y)
&\ge b(\PrefixB, \x, \y) - b(\Prefix, \x, \y) - \omega - \Omega,
\label{eq:similar}
\end{align}
where~$b(\Prefix, \x, \y)$ and~$b(\PrefixB, \x, \y)$ are the
objective lower bounds of~$\RL$ and~$\RLB$, respectively.
\end{theorem}

\begin{proof}
We begin by defining four related rule lists.
%
First, let ${\RL = (\Prefix, \Labels, \Default, K)}$
be a rule list with prefix ${\Prefix = (p_1, \dots, p_K)}$
and labels ${\Labels = (q_1, \dots, q_K)}$.
%
Second, let ${\RLB = (\PrefixB, \LabelsB, \DefaultB, \kappa)}$
be a rule list with prefix ${\PrefixB = (P_1, \dots, P_\kappa)}$
and labels ${\LabelsB = (Q_1, \dots, Q_\kappa)}$.
%
Define~$\omega$ as in~\eqref{eq:omega}
and~$\Omega$ as in~\eqref{eq:big-omega},
and require that~${\omega, \Omega \le \Reg}$.
%
Third, let ${\RL' = (\Prefix', \Labels', \Default', K') \in}$
${\StartsWith(\Prefix)}$ be any rule list
whose prefix starts with~$\Prefix$, such that~${K' \ge K}$.
%
Denote the prefix and labels of~$\RL'$ by
${\Prefix' = (p_1, \dots, p_K, p_{K+1}, \dots, p_{K'})}$
and ${\Labels = (q_1, \dots, q_{K'})}$,
respectively.
%
Finally, define
${\RLB' = (\PrefixB', \LabelsB', \DefaultB', \kappa') \in \StartsWith(\PrefixB)}$
to be the `analogous' rule list, \ie whose prefix
${\PrefixB' = (P_1, \dots, P_\kappa, P_{\kappa+1}, \dots, P_{\kappa'})
= (P_1, \dots, P_\kappa, p_{K+1}, \dots, p_{K'})}$
starts with~$\PrefixB$ and ends with the same ${K'-K}$
antecedents as~$\Prefix'$.
%
Let ${\LabelsB' = (Q_1, \dots, Q_{\kappa'})}$
denote the labels of~$\RLB'$.

[Expand on this.]
%Suppose that the lower bounds of~$\Prefix$ and~$\PrefixB$
%obey ${b(\Prefix, \x, \y) < b(\PrefixB, \x, \y)}$.
%
The smallest possible objective for~$\RLB'$, in relation
to the objective of~$\RL'$, reflects both the difference
between the objective lower bounds of~$\RLB$ and~$\RL$
and the largest possible discrepancy between the
objectives of~$\RL'$ and~$\RLB'$.
%
The latter would occur if~$\RL'$ misclassified all the data
corresponding to both~$\omega$ and~$\Omega$ while~$\RLB'$
correctly classified this same data, thus
\begin{align}
\Obj(\RLB', \x, \y) \ge \Obj(\RL', \x, \y)
  + b(\PrefixB, \x, \y) - b(\Prefix, \x, \y) - \omega - \Omega.
\label{eq:similar-analogous}
\end{align}
%

Now let~$\RLB^*$ be an optimal rule list with prefix
constrained to start with~$\PrefixB$,
\begin{align}
\RLB^* \in \argmin_{\RLB^\dagger \in \StartsWith(\PrefixB)} \Obj(\RLB^\dagger, \x, \y),
\end{align}
and let~$\kappa^*$ be the length of~$\RLB^*$.
%
Also let~$\RL^*$ be the analogous $K^*$-rule list whose prefix
starts with~$\Prefix$ and ends with the same~${\kappa^* - \kappa}$
antecedents as~$\RLB^*$, where~${K^* = K + \kappa^* - \kappa}$.
%
By~\eqref{eq:similar-analogous},
\begin{align}
\min_{\RLB^\dagger \in \StartsWith(\PrefixB)} \Obj(\RLB^\dagger, \x, \y)
= \Obj(\RLB^*, \x, \y) &\ge \Obj(\RL^*, \x, \y)
  + b(\PrefixB, \x, \y) - b(\Prefix, \x, \y) - \omega - \Omega \nn \\
&\ge \min_{\RL^\dagger \in \StartsWith(\Prefix)} \Obj(\RL^\dagger, \x, \y)
  + b(\PrefixB, \x, \y) - b(\Prefix, \x, \y) - \omega - \Omega.
\end{align}

\end{proof}

Theorem~\ref{thm:similar} implies that if prefixes~$\Prefix$
and~$\PrefixB$ are similar, and we know the optimal objective
of rule lists starting with~$\Prefix$, then
\begin{align}
\min_{\RLB' \in \StartsWith(\PrefixB)} \Obj(\RLB', \x, \y)
&\ge \min_{\RL' \in \StartsWith(\Prefix)} \Obj(\RL', \x, \y)
+ b(\PrefixB, \x, \y) - b(\Prefix, \x, \y) - \chi \nn \\
&\ge \CurrentObj + b(\PrefixB, \x, \y) - b(\Prefix, \x, \y) - \chi,
\end{align}
where~$\CurrentObj$ is the current best objective,
and~$\chi$ is the normalized support of the set of data captured
either exclusively by~$\Prefix$ or exclusively by~$\PrefixB$.
%
It follows that
\begin{align}
\min_{\RLB' \in \StartsWith(\PrefixB)} \Obj(\RLB', \x, \y)
\ge \CurrentObj + b(\PrefixB, \x, \y) - b(\Prefix, \x, \y) - \chi \ge \CurrentObj
\end{align}
if ${b(\PrefixB, \x, \y) - b(\Prefix, \x, \y) \ge \chi}$.
%
To conclude, we summarize this result and combine it with
our notion of lookahead from Lemma~\ref{lemma:lookahead}.
%
During branch-and-bound execution, if we demonstrate that
${\min_{\RL' \in \StartsWith(\Prefix)} \Obj(\RL', \x, \y) \ge \CurrentObj}$,
then we can prune all prefixes that start with any
prefix~$\PrefixB'$ in the following set:
\begin{align}
\left\{ \PrefixB' : b(\PrefixB', \x, \y) + \Reg - b(\Prefix, \x, \y) \ge
\frac{1}{N} \sum_{n=1}^N \Cap(x_n, \Prefix) \oplus \Cap(x_n, \PrefixB') \right\}.
\end{align}

\begin{comment}
To demonstrate~\eqref{eq:similar-analogous-optimal},
we consider two separate scenarios.
%
In the first scenario, prefixes~$\Prefix$ and~$\PrefixB$
capture the same data, therefore ${\omega = \Omega = 0}$.
%
By claim~\eqref{eq:equiv-analogous-optimal},
which we justify in the proof of Theorem~\ref{thm:equivalent},
${\RLB^* \in \argmin_{\RLB^\dagger \in \StartsWith(\PrefixB)} \Obj(\RLB^\dagger, \x, \y)}$,
thus
\begin{align}
\Obj(\RLB^*, \x, \y)
  = \min_{\RLB^\dagger \in \StartsWith(\PrefixB)} \Obj(\RLB^\dagger, \x, \y)
\le \min_{\RLB^\dagger \in \StartsWith(\PrefixB)}
  \Obj(\RLB^\dagger, \x, \y) + \omega + \Omega.
\end{align}

In the second scenario, prefixes~$\Prefix$ and~$\PrefixB$
do not capture the same data, thus either~${\omega \neq 0}$,
or~${\omega \neq 0}$, or both.
%
Consequently, prefixes~$\Prefix$ and~$\PrefixB$
are not composed of the same antecedents.
%
Define~${\phi = \{p_k : (p_k \in \Prefix) \wedge (p_k \notin \PrefixB)\}}$
to be the set of antecedents in~$\Prefix$ that are not in~$\PrefixB$,
and define~${\Phi = \{P_k : (P_k \in \PrefixB) \wedge (P_k \notin \Prefix)\}}$
to be the set of antecedents in~$\PrefixB$ that are not in~$\Prefix$;
either~${\phi \neq \emptyset}$, or~${\Phi \neq \emptyset}$, or both.

Suppose~${\phi \neq \emptyset}$, and let~${p \in \phi}$
be an antecedent in~$\phi$.
%
It follows that there exists a subset of rule lists
in~$\StartsWith(\PrefixB)$ that do not have analogues
in~$\StartsWith(\Prefix)$.
%
Let~${\RLB'' \in \StartsWith(\PrefixB)}$ be such a rule list,
such that its prefix ${\PrefixB'' = (P_1, \dots, P_\kappa, \dots, p, \dots)}$
starts with~$\PrefixB$ and contains~$p$ among its remaining antecedents.
%
Since~$p$ captures a subset of the data that~$\Prefix$ captures,
and~$\PrefixB$ captures a similar set of data as~$\Prefix$, it follows
that~$p$ in~$\PrefixB''$ can capture at most a small set of data
no larger than~$\omega$:
\begin{align}
\frac{1}{N} \sum_{n=1}^N \Cap(x_n, p \given \PrefixB'')
&\le \frac{1}{N} \sum_{n=1}^N \neg\, \Cap(x_n, \PrefixB) \wedge \Cap(x_n, p) \nn \\
&\le \frac{1}{N} \sum_{n=1}^N \neg\, \Cap(x_n, \PrefixB) \wedge \Cap(x_n, \Prefix)
= \omega \le \Reg.
\end{align}
By Theorem~\ref{thm:min-capture}, antecedent~$p$ has insufficient
support in~$\RLB''$, and thus~$\RLB''$ cannot be optimal, \ie
${\RLB'' \notin}$ ${\argmin_{\RLB^\dagger \in \StartsWith(\PrefixB)} \Obj(\RLB^\dagger, \x, \y)}$.
%
Similarly, if~${\Phi \neq \emptyset}$
and~${P \in \Phi}$, and~${\RL'' \in \StartsWith(\Prefix)}$
is any rule list whose prefix starts with~$\Prefix$
and contains antecedent~$P$, then~$P$ in~$\Prefix''$
only has normalized support
\begin{align}
\frac{1}{N} \sum_{n=1}^N \Cap(x_n, P \given \Prefix'')
&\le \frac{1}{N} \sum_{n=1}^N \neg\, \Cap(x_n, \Prefix) \wedge \Cap(x_n, P) \nn \\
&\le \frac{1}{N} \sum_{n=1}^N \neg\, \Cap(x_n, \Prefix) \wedge \Cap(x_n, \PrefixB)
= \Omega \le \Reg.
\end{align}
By Theorem~\ref{thm:min-capture}, antecedent~$P$ has insufficient
support in~$\RL''$, and thus~$\RL''$ can't be optimal, \ie ${\RL'' \notin}$
${\argmin_{\RL^\dagger \in \StartsWith(\Prefix)} \Obj(\RL^\dagger, \x, \y)}$.

To finish justifying claim~\eqref{eq:similar-analogous-optimal}
for the second scenario, first define
\begin{align}
\tau(\Prefix, \Phi) \equiv
  \{\RL'' = (\Prefix'', \Labels'', \Default'', K'') :
    \RL'' \in \StartsWith(\Prefix) \textnormal{ and }
    p_k \notin \Phi, \forall p_k \in \Prefix''\} \subset \StartsWith(\Prefix)
\end{align}
to be the set of all rule lists whose prefixes start with~$\Prefix$
and don't contain any antecedents in~$\Phi$.
%
Now, recognize that the optimal prefixes in~$\tau(\Prefix, \Phi)$
and~$\StartsWith(\Prefix)$ are the same, \ie
\begin{align}
\argmin_{\RL^\dagger \in \tau(\Prefix, \Phi)} \Obj(\RL^\dagger, \x, \y)
= \argmin_{\RL^\dagger \in \StartsWith(\Prefix)} \Obj(\RL^\dagger, \x, \y),
\end{align}
and similarly, the optimal prefixes in~$\tau(\PrefixB, \phi)$
and~$\StartsWith(\PrefixB)$ are the same, \ie
\begin{align}
\argmin_{\RLB^\dagger \in \tau(\PrefixB, \phi)} \Obj(\RLB^\dagger, \x, \y)
= \argmin_{\RLB^\dagger \in \StartsWith(\PrefixB)} \Obj(\RLB^\dagger, \x, \y).
\end{align}
We have shown that every~${\RL'' \in \tau(\Prefix, \Phi)}$
has a direct analogue~${\RLB'' \in \tau(\PrefixB, \phi)}$,
such that~$\RL''$ and~$\RLB''$ obey~\eqref{eq:similar-analogous},
and vice versa.

By~\eqref{eq:similar-analogous},
\begin{align}
\Obj(\RLB^*, \x, \y) &\ge \Obj(\RL^*, \x, \y)
  - \left( b(\PrefixB, \x, \y) - b(\Prefix, \x, \y) \right) - \omega - \Omega \nn \\
&= \min_{\RL^\dagger \in \StartsWith(\Prefix)} \Obj(\RL^\dagger, \x, \y)
  - \left( b(\PrefixB, \x, \y) - b(\Prefix, \x, \y) \right) - \omega - \Omega.
\label{eq:sa-1}
\end{align}

[Explain this.] Combining~\eqref{eq:sa-1} and~\eqref{eq:sa-2} gives~\eqref{eq:similar-analogous-optimal}.

\end{proof}

\begin{theorem}[Very similar support bound]
\label{thm:very-similar}
Define ${\StartsWith(\Prefix) = }$
${\{(\Prefix', \Labels', \Default', K') : \Prefix' \textnormal{ starts with } \Prefix \}}$
to be the set of all rule lists whose prefixes start with~$\Prefix$.
%
Let ${\Prefix = (p_1, \dots, p_K)}$ and
${\PrefixB = (P_1, \dots, P_{\kappa})}$ be prefixes
that capture the same data, \ie
\begin{align}
\{x_n : \Cap(x_n, \Prefix)\} = \{x_n : \Cap(x_n, \PrefixB)\}.
\end{align}
Now let ${\Prefix' = (p_1, \dots, p_K, p_{K+1})}$ be a prefix
that starts with~$\Prefix$ and ends with antecedent~$p_{K+1}$,
and let ${\PrefixB' = (P_1, \dots, P_{\kappa}, P_{\kappa + 1})}$
be a prefix that starts with~$\PrefixB$ and ends with
antecedent~$P_{\kappa + 1}$, such that~$p_{K+1}$
and~$P_{\kappa + 1}$ have nearly the same support in~$\Prefix'$
and~$\PrefixB'$, respectively.
%
Specifically, define~$\omega$ to be the normalized support
of data captured by~$p_{K + 1}$ in~$\Prefix'$ and not
captured by~$P_{\kappa + 1}$ in~$\PrefixB'$,
and let us require that~${\omega \le \Reg}$, \ie
\begin{align}
\omega \equiv \frac{1}{N} \sum_{n=1}^N
  \neg\, \Cap(x_n, P_{\kappa+1} \given \PrefixB')
  \wedge \Cap(x_n, p_{K+1} \given \Prefix') \le \Reg,
\end{align}
where~$\Reg$ is the regularization parameter.
%
Similarly, define~$\Omega$ to be the normalized support
of data captured by~$P_{\kappa + 1}$ in~$\PrefixB'$ and not
captured by~$p_{K + 1}$ in~$\Prefix'$,
and let us require that~${\Omega \le \Reg}$, \ie
\begin{align}
\Omega \equiv \frac{1}{N} \sum_{n=1}^N
  \neg\, \Cap(x_n, p_{K+1} \given \Prefix')
  \wedge \Cap(x_n, P_{\kappa+1} \given \PrefixB') \le \Reg.
\end{align}
The difference between the objective of the optimal rule list
in~$\StartsWith(\Prefix')$ and that in~$\StartsWith(\PrefixB')$
is at most~$2\Reg$:
\begin{align}
\left | \min_{\RL^\dagger \in \StartsWith(\Prefix')} \Obj(\RL^\dagger, \x, \y)
  - \min_{\RLB^\dagger \in \StartsWith(\PrefixB')} \Obj(\RLB^\dagger, \x, \y)
  \right | \le 2 \Reg.
\label{eq:similar}
\end{align}
\end{theorem}

\begin{proof}
\dots
\end{proof}

Suppose prefixes~$\cal P$ and~$\cal Q$ capture the same data,
and now derive~$\cal P'$ from~$\cal P$ by appending antecedent~$p$
and derive~$\cal Q'$ from~$\cal Q$ by appending antecedent~$q$.
%
Suppose further that~$p$ and~$q$ capture nearly the same data, except that
they exclusively capture data~$x_p$ and~$x_q$ in their respective contexts,
such that the normalized support of~$x_p$ and of~$x_q$ are each bounded by
the regularization parameter: ${s(x_p), s(x_q) < c}$.
%
Our minimum support bound~\eqref{eq:min-capture} implies
that~$p$ would never be placed below~$\cal Q'$ nor~$q$ below~$\cal P'$.

Extensions of~$\cal P'$ and~$\cal Q'$ will behave similarly.
%
The largest difference would occur if a rule list starting with~$\cal P'$
misclassified all of~$x_p$ and~$x_q$, while the analogous rule list starting
with~$\cal Q'$ correctly classified all these data, or vice versa,
yielding a difference between objectives bounded by~$2c$.
%
Let~$\cal P^*$ and~$\cal Q^*$ be the optimal prefixes
starting with~$\cal P'$ and~$\cal Q'$, respectively.
%
Note that~$\cal P^*$ and~$\cal Q^*$ need not be derived from~$\cal P'$ and~$\cal Q'$
via analogous extensions.
%
If we know~$\cal P^*$, then we can avoid evaluating \emph{any} extensions of~$\cal Q'$ if
\begin{align}
\Obj(\Prefix^*) - 2 c \ge \Obj^*,
\end{align}
where~$\Obj^*$ is the best known objective, since the left had expression
provides a lower bound on~$\Obj(\cal{Q}^*)$.
\end{comment}

% ELA would like to generalize this to any two prefixes that capture similar data

\subsection{Equivalence of rule lists when rules commute}

\begin{theorem}
\end{theorem}

\begin{proof}
\end{proof}

\end{arxiv}

\begin{comment}
If two rules~$A$ and~$B$ capture disjoint subsets of data,
then they commute globally in the sense that any rule list where~$A$ and~$B$ are
adjacent is equivalent to another rule list where~$A$ and~$B$ swap positions.
%
More generally, a rule list containing possibly multiple, possibly overlapping
pairs of commuting rules is equivalent to any other rule list that can be generated
by swapping one or more such pairs of rules.
%
We can avoid evaluating multiple such equivalent rule lists by eliminating all but one.
%
%We achieve this by restricting how we grow a prefix (by appending a rule), as informed by a hash map cdict that maps each rule R_i to a set of rules S_i = {R_j} such that R_i commutes with every R_j in S_i and j > i. When growing a prefix that ends with rule R_i, we only append a rule R_j if it is not in S_i.

\begin{itemize}
\item other permutation bounds
\end{itemize}
\end{comment}

%For any prefix P, we can write down a lower bound on its permutations' objectives.  (I'm not talking about longer prefixes, just permutations of the same length, call it L.)   An easy bound comes from assuming that there is a permutation that makes no mistakes on captured data, so the only contributions to the objective come from the mistakes of the default rule (call this m) and c*L.  All the permutations capture the same data, so m is constant and this gives

%min objective( any permutation of P ) >= m/n + c*L

%Actually we can make this tighter if we know lower bound information from permutations of sub-prefixes.  For example, the length (L-1) "sub-prefixes" of P generate L permutation groups.  We can add the minimum number of mistakes (on captured data) made by prefixes over all of these groups to our bound, call this k

%min objective( any permutation of P ) >= k/n + m/n + c*L

%We only know k if we've already evaluated the length (L-1) permutations -- we do know it in the breadth-first setting.  (This idea could be generalized to shorter sub-prefixes if desired, e.g., for policies other than breadth first.)

\subsection{Equivalent points bound}
\label{sec:identical}

The bounds in this section quantify the following:
%
If multiple observations that are not captured by a prefix~$\Prefix$
have identical features and opposite labels, then no rule list that
starts with~$\Prefix$ can correctly classify all these observations.
%
For each set of such observations, the number of mistakes is at least
the number of observations with the minority label within the set.

Consider a dataset~${\{(x_n, y_n)\}_{n=1}^N}$ and a set of antecedents~${\{s_m\}_{m=1}^M}$.
%
Define distinct datapoints to be equivalent if they are captured by
exactly the same antecedents, \ie $x_i \neq x_j$ are equivalent if
\begin{align}
\frac{1}{M} \sum_{m=1}^M \one [ \Cap(x_i, s_m) = \Cap(x_j, s_m) ] = 1.
\end{align}
Notice that we can partition a dataset into sets of equivalent points;
let~${\{e_u\}_{u=1}^U}$ enumerate these sets.
%
Now define~$\theta(e_u)$ to be the normalized support of the minority
class label with respect to set~$e_u$, \eg let
\begin{arxiv}
\begin{align}
{e_u = \{x_n : \one [ \Cap(x_n, s_m) = \Cap(x_i, s_m) ] \}},
\end{align}
\end{arxiv}
\begin{kdd}
${e_u = \{x_n : \one [ \Cap(x_n, s_m) = \Cap(x_i, s_m) ] \}}$,
\end{kdd}
and let~$q_u$ be the minority class label among points in~$e_u$, then
\begin{align}
\theta(e_u) = \frac{1}{N} \sum_{n=1}^N \one [ x_n \in e_u ] \wedge \one [ y_n = q_u ].
\end{align}

The existence of equivalent points sets with non-singleton support
yields a tighter objective lower bound that we can combine with our other bounds;
as our experiments demonstrate~(\S\ref{sec:experiments}),
the practical consequences can be dramatic.
%
First, for intuition, we present a global bound in
Proposition~\ref{prop:identical}; next, we explicitly integrate
this bound into our framework in Theorem~\ref{thm:identical}.

\begin{proposition}[Global equivalent points bound]
\label{prop:identical}
Let~$\RL$ be any rule list, then
\begin{arxiv}
\begin{align}
\Obj(\RL, \x, \y) \ge \sum_{u=1}^U \theta(e_u).
\end{align}
\end{arxiv}
\begin{kdd}
$\Obj(\RL, \x, \y) \ge \sum_{u=1}^U \theta(e_u)$.
\end{kdd}
\end{proposition}

Now, recall that to obtain our lower bound~${b(\Prefix, \x, \y)}$
in~\eqref{eq:lower-bound}, we simply deleted the
default rule misclassification error~$\Loss_0(\Prefix, \Default, \x, \y)$
from the objective~${\Obj(\RL, \x, \y)}$.
%
Theorem~\ref{thm:identical} obtains a tighter objective lower bound
via a tighter lower bound~${0 \le b_0(\Prefix, \x, \y) \le}$
$\Loss_0(\Prefix, \Default, \x, \y)$ on the default rule misclassification error.

\begin{theorem}[Equivalent points bound]
\label{thm:identical}
Let~$\RL$ be a rule list with prefix~$\Prefix$
and lower bound~${b(\Prefix, \x, \y)}$,
then for any rule list~${\RL' \in \StartsWith(\RL)}$
whose prefix~$\Prefix$ starts with~$\Prefix$,
\begin{arxiv}
\begin{align}
\Obj(\RL', \x, \y) \ge b(\Prefix, \x, \y) + b_0(\Prefix, \x, \y),
\label{eq:identical}
\end{align}
where
\end{arxiv}
\begin{kdd}
\begin{align}
\Obj(\RL', \x, \y) \ge b(\Prefix, \x, \y) + b_0(\Prefix, \x, \y), \quad \text{where}
\end{align}
\end{kdd}
\begin{arxiv}
\begin{align}
b_0(\Prefix, \x, \y) = \frac{1}{N} \sum_{u=1}^U \sum_{n=1}^N
    \neg\, \Cap(x_n, \Prefix) \wedge \one [ x_n \in e_u ] \wedge \one [ y_n = q_u ].
\end{align}
\end{arxiv}
\begin{kdd}
\begin{align}
b_0(\Prefix, \x, \y) = \frac{1}{N} \sum_{u=1}^U \sum_{n=1}^N
    \neg\, \Cap(x_n, \Prefix) \wedge \one [ x_n \in e_u ] \wedge \one [ y_n = q_u ]. \nn
\end{align}
\end{kdd}
\end{theorem}
