\begin{comment}
Notes:
\begin{itemize}

\item Measurements over multiple executions (corresponding to ten folds)
should include standard deviations

\item Most figures will be generated with one particular big dataset
\end{itemize}

Probably want to integrate some measurements of the permutation map's effects
(from Margo's email):
\begin{itemize}
\item How many times do we look up an item in the hash map that is not in the tree?

\item When we do look up an item, how many times do we end up discarding
      something we would have been unable to discard had we not done this.

\item How is the hash map growing over time?

\item How is the tree growing over time?
\end{itemize}

Figures:
\begin{itemize}

\item Figure~\ref{fig:comparison}, Comparison with other methods:
Test error for us and a few other algorithms
(CART, C4.5, CBA, CMAR/CPAR, C5.0, Ripper, \dots),
as a function of sparsity (number of rules) over 10 folds, for one big dataset (box plots).
Also report algorithm runtimes (mean $\pm$ standard deviation over 10 folds).
Also compare accuracies to random forests and boosted decision trees.

\item Figure~\ref{fig:regularization}
Missing:  Test error as a function of regularization and sparsity
(number of rules) as a function of regularization, over 10 folds,
for one big dataset.

\item Figure~\ref{fig:ablation},  Ablation experiment:
Show the effect of each ``piece'' at a time,
run X without each in turn and show the difference in either
quality of solution or runtime or amount of memory, size of cache or queue,
where X is a specific implementation
(meaning a specific scheduling policy and node type)

\item Figure~\ref{fig:scheduling-policy}
Missing:  Some sort of comparison of different scheduling policies

\item Figure~\ref{fig:queue-cache-size-insertions} (top),
Size of cache and queue data structures as a function of wall clock time

\item Figure~\ref{fig:queue-cache-size-insertions} (bottom),
Cumulative number of things placed in queue over time

\item Figure~\ref{fig:objective}, Objective value over time,
with horizontal lines and x-ticks for CART and C4.5
%-- this also gives an upper bound on the remaining search space --
%and also, possibly, the minimum lower bound in the queue over time

\item Figure~\ref{fig:search-space},
Fraction of search space we've proven we've handled

\item Figure~\ref{fig:max-length},
Max prefix length over time (computed from objective value);

\item Figure~\ref{fig:prefix-length},
Best prefix length over time

\item Some characterization of the number of solutions close to optimal
(plot number of suboptimal solutions vs amount of suboptimality,
removing permutations)

\item Some example rule lists to show how interpretable they are.
Potentially we could display equally optimal rule lists that look
very different from each other.

\end{itemize}
\end{comment}
