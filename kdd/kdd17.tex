\documentclass[sigconf]{acmart}

\usepackage{booktabs} % For formal tables


% Copyright
%\setcopyright{none}
%\setcopyright{acmcopyright}
%\setcopyright{acmlicensed}
\setcopyright{rightsretained}
%\setcopyright{usgov}
%\setcopyright{usgovmixed}
%\setcopyright{cagov}
%\setcopyright{cagovmixed}


% DOI
\acmDOI{10.475/123_4}

% ISBN
\acmISBN{123-4567-24-567/08/06}

%Conference
\acmConference[KDD'17]{ACM Woodstock conference}{August 13 - 17, 2017}{Halifax, Nova Scotia, Canada} 
\acmYear{2017}
\copyrightyear{2017}

\acmPrice{15.00}


\begin{document}
\title{Learning rule lists with branch and bound}
\titlenote{Produces the permission block, and
  copyright information}
\subtitle{Extended Abstract}
\subtitlenote{The full version of the author's guide is available as
  \texttt{acmart.pdf} document}


\author{Elaine Angelino}
%\authornote{Dr.~Trovato insisted his name be first.}
%\orcid{1234-5678-9012}
\affiliation{%
  \institution{EECS, UC Berkeley}
  %\streetaddress{}
  \city{Berkeley} 
  \state{CA} 
  \postcode{94720}
}
\email{elaine@eecs.berkeley.edu}

\author{Nicholas Larus-Stone}
%\authornote{The secretary disavows any knowledge of this author's actions.}
\affiliation{%
  \institution{SEAS, Harvard University}
  %\streetaddress{}
  \city{Cambridge} 
  \state{MA} 
  \postcode{02138}
}
\email{nlarusstone@college.harvard.edu}

\author{Daniel Alabi}
%\authornote{The secretary disavows any knowledge of this author's actions.}
\affiliation{%
  \institution{SEAS, Harvard University}
  %\streetaddress{}
  \city{Cambridge} 
  \state{MA} 
  \postcode{02138}
}
\email{alabid@g.harvard.edu}

\author{Margo Seltzer}
%\authornote{The secretary disavows any knowledge of this author's actions.}
\affiliation{%
  \institution{SEAS, Harvard University}
  %\streetaddress{}
  \city{Cambridge} 
  \state{MA} 
  \postcode{02138}
}
\email{margo@eecs.harvard.edu}

\author{Cynthia Rudin}
\affiliation{%
  \institution{Duke University}
  \city{Durham}
  \state{NC} 
  \postcode{27708}}
\email{cynthia@cs.duke.edu}

% The default list of authors is too long for headers}
\renewcommand{\shortauthors}{E. Angelino et al.}


\begin{abstract}
This paper provides a sample of a \LaTeX\ document which conforms,
somewhat loosely, to the formatting guidelines for
ACM SIG Proceedings\footnote{This is an abstract footnote}. 
\end{abstract}

%
% The code below should be generated by the tool at
% http://dl.acm.org/ccs.cfm
% Please copy and paste the code instead of the example below. 
%
\begin{CCSXML}
<ccs2012>
 <concept>
  <concept_id>10010520.10010553.10010562</concept_id>
  <concept_desc>Computer systems organization~Embedded systems</concept_desc>
  <concept_significance>500</concept_significance>
 </concept>
 <concept>
  <concept_id>10010520.10010575.10010755</concept_id>
  <concept_desc>Computer systems organization~Redundancy</concept_desc>
  <concept_significance>300</concept_significance>
 </concept>
 <concept>
  <concept_id>10010520.10010553.10010554</concept_id>
  <concept_desc>Computer systems organization~Robotics</concept_desc>
  <concept_significance>100</concept_significance>
 </concept>
 <concept>
  <concept_id>10003033.10003083.10003095</concept_id>
  <concept_desc>Networks~Network reliability</concept_desc>
  <concept_significance>100</concept_significance>
 </concept>
</ccs2012>  
\end{CCSXML}

\ccsdesc[500]{Computer systems organization~Embedded systems}
\ccsdesc[300]{Computer systems organization~Redundancy}
\ccsdesc{Computer systems organization~Robotics}
\ccsdesc[100]{Networks~Network reliability}

% We no longer use \terms command
%\terms{Theory}

\keywords{ACM proceedings, \LaTeX, text tagging}

\maketitle

\documentclass[aoas,preprint]{imsart}
\usepackage{fullpage}
\setattribute{journal}{name}{}
\usepackage[usenames,dvipsnames,svgnames,table]{xcolor}

\usepackage{graphicx,verbatim}
\usepackage[round]{natbib}
\usepackage{url}
\usepackage{amsmath,amssymb,amsthm,amsfonts}
\usepackage{algorithm}
\usepackage{algpseudocode}
\usepackage{todonotes}
\usepackage{subfig}
\usepackage{dsfont}
\usepackage{listings}
\usepackage{comment}

\usepackage{tikz}
\usetikzlibrary{arrows}

\newcommand{\eanote}[1]{{\color{magenta} (EA) #1}}
\newcommand{\red}[1]{{\color{red} #1}}
\newcommand{\yellow}[1]{{\color{yellow} #1}}
\newcommand{\green}[1]{{\color{green} #1}}
\newcommand{\eat}[1]{ }

\def\ie{{\it i.e.},~}
\def\eg{{\it e.g.},~}
\def\etal{{\it et al.}~}

\def\E{\mathbb{E}}
\def\P{\mathbb{P}}
\def\Var{\mbox{Var}}
\def\Unif{\mbox{Unif}}
\def\Normal{\mbox{Normal}}
\def\reals{\mathbb{R}}
\def\ints{\mathbb{Z}}
\def\one{\mathds{1}}
\def\Normal{\mathrm{Normal}}
\def\X{{\mathcal X}}
\def\Y{{\mathcal Y}}
\def\RL{{\mathcal R}}
\def\N{{\mathcal N}}
\def\Prefix{{\mathcal P}}
\def\RuleSet{{\mathcal S}}
\newcommand{\x}{\mathbf{x}}
\newcommand{\y}{\mathbf{y}}

\newcommand{\eins}{\mbox{$1 \hspace{-1.0mm} {\bf l}$}}
\newcommand{\A}{\mathcal{A}}
\newcommand{\Ac}{\mathcal{A}^c}
\newcommand{\T}[3]{T_{#1}(#2 \leftarrow #3)}
\def\reals{\mathbb{R}}
\def\one{\mathds{1}}

\newtheorem{lemma}{Lemma}
\newtheorem{theorem}{Theorem}
\newtheorem{remark}{Remark}
\newtheorem{definition}{Definition}
\newtheorem{proposition}{Proposition}
\newtheorem{claim}{Claim}

\newcommand{\noi}{\noindent}
\newcommand{\nn}{\nonumber}
\newcommand{\be}{\begin{equation}}
\newcommand{\ee}{\end{equation}}
\newcommand{\bea}{\begin{eqnarray}}
\newcommand{\eea}{\end{eqnarray}}
\newcommand{\erf}{\text{erf}}
\newcommand{\bits}[1]{\texttt{#1}}

\newcommand{\given}{\,|\,}

\frenchspacing
\hyphenation{speed-up}

\begin{document}

\section{Introduction}

As machine learning continues to grow in importance for socially-important decisions, the interpretability of predictive models remains a crucial problem. Our aim is to build models that are highly predictive but in which each step of the model's decision making process can also be investigated by humans. To achieve this, we use rule lists, also known as decision lists, which are lists composed of if-then statements. This structure allows for predictive models that also can be easily interpreted; the rules give a reason for each prediction.

The problem of how to construct rule lists, or more generally, decision trees, has been in existence at least 30 years \cite{rivest:1987,CART,C5.0}. The vast majority of approaches use greedy splitting techniques \cite{rivest:1987,CART,C5.0,etc}. More modern techniques have used Bayesian analysis, either to find a locally optimal solution \cite{BART} or to actually explore the search space \citep{LethamRuMcMa15, YangRuSe16}. These more modern approaches achieve high accuracy while also managing to run reasonably quickly. However, despite the apparent accuracy of the rule lists generated by these algorithms, there is no way to determine if the generated rule list is optimal or how close to optimal the rule lists is.

Optimality is important because there are societal implications for a lack of optimality. The Pro-Publica article on the COMPAS recidivism prediction tool \citep{LarsonMaKiAn16} is one example. It highlights a case where a black-box, proprietary predictive model is being used for recidivism prediction. \citep{LarsonMaKiAn16}  show that the COMPAS scores are racially biased, but since the model is not transparent, no one (outside of the creators of COMPAS) can determine the reason or extent of the bias. Nor can anyone determine the true reasons for any particular prediction. It was determined that a transparent model would not be sufficiently accurate for recidivism prediction, thus a more accurate, black box model would have to suffice. We wondered whether there was indeed no possible transparent model that would suffice. To determine the answer to that question, one would need to solve a computationally hard problem, namely to find a transparent model that is actually optimal, with a certificate of optimality, among a particular pre-determined class of models. That way one could say, with certainty, whether a transparent model (from this class of models) with sufficient accuracy exists for this problem, before resorting to black box models.

In our framework, we consider the class of rule lists created from pre-mined frequent itemsets. The rule list must be assembled from these itemsets to minimize a regularized risk functional, $R$. This is a hard discrete optimization problem. A brute force solution to find the rule list that minimizes $R$ would be computationally prohibitive due to the exponential number of possible rule lists. This, however, is a worst case bound that is not realized in practical settings. For realistic cases, it is possible to solve fairly large cases of this problem to optimality, with the careful use of algorithms, data structures, and bit-vector manipulation.

We develop specialized tools from the field of discrete optimization and artificial intelligence, and in particular, a special branch-and-cut algorithm, called "??", that provides (1) the optimal solution, (2) near-optimal solutions, (3) a certificate of optimality, and if optimality is not achieved, the distance between the current solution and optimality. Because of the certificate of optimality, this method can also be used to investigate how close other models (e.g., models provided by greedy algorithms) are to optimality. In particular, we can investigate if the rule lists from probabilistic approaches are nearly optimal or whether those approaches sacrifice too much accuracy to gain speed for constructing the model.

Within its branch-and-cut procedure, ??? maintains an upper bound on the maximum value of $R$ that each incomplete rule list can achieve. This allows it to prune an incomplete rule list (and every possible extension) if the bound is worse than the accuracy of the best rule list that we've already looked at. The use of careful bounding techniques leads to massive pruning of the search space of potential rule lists. We continue to consider incomplete and complete rule lists until we have either looked at every rule list or eliminated it from consideration. Thus, the end of execution leaves us with the optimal rule list, the close-to-optimal rule lists, and a certificate of optimality.

The efficacy of ??? depends on how much of the search space our bounds allow us to prune. The upper bound on $R$ must thus be as tight as reasonably possible. The bound we maintain throughout the calculation is a minimum of several bounds, that come in three categories. The first category of bounds are those intrinsic to the rules themselves. This category includes bounds stating that each rule must capture sufficient data; if not, the rule list is provably non-optimal. The second type of bound compares an upper bound on the value of $R$ to that of the current best solution. This allows us to exclude parts of the search space that could never be better than our current solution. Finally, our last type of bound is based on comparing incomplete rule lists that capture the same data and pursuing only the more accurate option. This last class of bounds is especially important -- without our use of a novel \textit{symmetry-aware map}, we are unable to solve most problems of reasonable scale. This symmetry-aware map keeps track of the best accuracy for all the permutations of a given incomplete rule list.

In order to keep track of all of these bounds for each rule list, we implemented a modified trie that we call a prefix tree. Each node in the prefix tree represents an individual rule; thus, each path in the tree represents a rule list where the final node in the path contains the metrics about that rule list. This tree structure facilities the use of multiple different selection algorithms including breadth-first search, a priority queue based on a custom curiosity function, and a stochastic selection process. In addition, we are able to limit the number of nodes in the tree and thereby achieve a way of tuning space-time tradeoffs in a robust manner. We propose that this tree structure is a useful way of organizing the generation of rule lists and could allow future implementations of ??? to be easily parallelized.

We evaluated BBRL??? on a number of publicly available datasets, and have made code for our algorithm and experiments publicly available. Our metric of success was 10-fold cross validated prediction accuracy on a subset of the data. These datasets involve hundreds of rules and hundreds or thousands of observations. BBRL??? is generally able to find the optimal rule list in a matter of seconds and certify it within about 10 minutes. We show that we are able to achieve better out-of-sample accuracy on these datasets than the popular greedy algorithms, CART or C5.0.

This algorithm is designed for solving large (not massive) problems, where interpretability and certifiable optimality is important. A key example of the type of problems this algorithm is useful for is recidivism prediction. Thus, we work on the dataset released by Northpointe, who designed the proprietary COMPAS algorithm. We show that, at least on this sample of the data, it is possible to produce an certifiably optimal interpretable rule list that achieves the same accuracy as a random forest applied to this dataset. We thus see no reason why a proprietary algorithm should be used for recidivism prediction.

\bibliographystyle{abbrvnat}
\bibliography{refs}

\end{document}

\documentclass[aoas,preprint]{imsart}
\usepackage{fullpage}
\setattribute{journal}{name}{}
\usepackage[usenames,dvipsnames,svgnames,table]{xcolor}

\usepackage{graphicx,verbatim}
\usepackage[round]{natbib}
\usepackage{url}
\usepackage{amsmath,amssymb,amsthm,amsfonts}
\usepackage{algorithm}
\usepackage{algpseudocode}
\usepackage{todonotes}
\usepackage{subfig}
\usepackage{dsfont}
\usepackage{listings}
\usepackage{comment}

\usepackage{tikz}
\usetikzlibrary{arrows}

\newcommand{\eanote}[1]{{\color{magenta} (EA) #1}}
\newcommand{\red}[1]{{\color{red} #1}}
\newcommand{\yellow}[1]{{\color{yellow} #1}}
\newcommand{\green}[1]{{\color{green} #1}}
\newcommand{\eat}[1]{ }

\def\ie{{\it i.e.},~}
\def\eg{{\it e.g.},~}
\def\etal{{\it et al.}~}

\def\E{\mathbb{E}}
\def\P{\mathbb{P}}
\def\Var{\mbox{Var}}
\def\Unif{\mbox{Unif}}
\def\Normal{\mbox{Normal}}
\def\reals{\mathbb{R}}
\def\ints{\mathbb{Z}}
\def\one{\mathds{1}}
\def\Normal{\mathrm{Normal}}
\def\X{{\mathcal X}}
\def\Y{{\mathcal Y}}
\def\RL{{\mathcal R}}
\def\N{{\mathcal N}}
\def\Prefix{{\mathcal P}}
\def\RuleSet{{\mathcal S}}
\newcommand{\x}{\mathbf{x}}
\newcommand{\y}{\mathbf{y}}

\newcommand{\eins}{\mbox{$1 \hspace{-1.0mm} {\bf l}$}}
\newcommand{\A}{\mathcal{A}}
\newcommand{\Ac}{\mathcal{A}^c}
\newcommand{\T}[3]{T_{#1}(#2 \leftarrow #3)}
\def\reals{\mathbb{R}}
\def\one{\mathds{1}}

\newtheorem{lemma}{Lemma}
\newtheorem{theorem}{Theorem}
\newtheorem{remark}{Remark}
\newtheorem{definition}{Definition}
\newtheorem{proposition}{Proposition}
\newtheorem{claim}{Claim}

\newcommand{\noi}{\noindent}
\newcommand{\nn}{\nonumber}
\newcommand{\be}{\begin{equation}}
\newcommand{\ee}{\end{equation}}
\newcommand{\bea}{\begin{eqnarray}}
\newcommand{\eea}{\end{eqnarray}}
\newcommand{\erf}{\text{erf}}
\newcommand{\bits}[1]{\texttt{#1}}

\newcommand{\given}{\,|\,}

\frenchspacing
\hyphenation{speed-up}

\begin{document}

\section{Related Work}


We will discuss related literature in several subfields.

\textit{Interpretable Models:} There is a growing interest in interpretable (transparent, comprehensible) models because of practical societal importance\citep[see][]{ruping2006learning,bratko1997machine,dawes1979robust,VellidoEtAl12,Giraud98,Holte93,Schmueli10,Huysmans11,Freitas14}. There are now regulations on algorithmic decision-making in the European Union on the ``right to an explanation" \citep{Goodman2016EU} that would legally require interpretability in predictions.


%@article{goodman2016eu,
%	Author = {Goodman, Bryce and Flaxman, Seth},
%	Date-Added = {2016-08-23 19:16:57 +0000},
%	Date-Modified = {2016-08-23 19:16:57 +0000},
%	Journal = {arXiv preprint arXiv:1606.08813},
%	Title = {EU regulations on algorithmic decision-making and a" right to explanation"},
%	Year = {2016}}
%

The body of work closest to ours is that on \textit{optimal decision tree modeling}. There is work starting in the late 1990's on building optimal decision trees using optimization techniques \citep[e.g.,][]{Bennett96optimaldecision,Auer95theoryand,dobkininduction}, continuing until the present \citep{e.g., farhangfar2008fast}. 
%@inproceedings{farhangfar2008fast,
%  title={A fast way to produce near-optimal fixed-depth decision trees},
%  author={Farhangfar, Alireza and Greiner, Russell and Zinkevich, Martin},
%  booktitle={Proceedings of the 10th international symposium on artificial intelligence and mathematics (ISAIM-2008)},
%  year={2008},
%  organization={Citeseer}
%}
A particularly interesting paper along these lines is that of \citet{NijssenFromont2010}, who created a ``bottom-up" way to form optimal decision trees. Their method performs an expensive search step, mining all possible leaves (rather than all possible rules), and using those leaves to form trees. Their method can lead to memory problems, but it is possible that these memory issues can be mitigated using the theorems in this paper. \footnote{There is no public version of their code for distribution as of this writing.} Another work close to ours is that of \citet{Garofalakis}, who introduce an algorithm to generate more interpretable decision trees by allowing constraints to be placed on the size of the decision tree. Like us, they use a branch-and-bound technique to constrain the size of the search space and limit the eventual size of the decision tree. During tree construction, they bound the possible Minimum Description Length (MDL) cost of every different split at a given node. If every split at that node is more expensive than the actual cost of the current subtree, then that node can be pruned. In this way, they were able to prune the tree while constructing it instead of just constructing the tree and then pruning at the end. \textcolor{red}{Hm?} However, even with the added bounds, this approach does not generally yield globally optimal decision trees because they constrained the number of nodes in the tree.

On the other end of the spectrum from optimal decision tree methods are \textit{greedy splitting and pruning} methods like CART \cite{} and C4.5 \cite{}. They do not perform exploration of the search space beyond greedy splitting.

There are \textit{Bayesian tree and rule list methods} that aim to explore the space of trees \cite{Dension:1998hl,Chipman:2002hc,Chipman10}, however, the space of trees of a given depth is much larger than the space of rule lists of that same level of depth, and the trees within these algorithms are grown in a top-down greedy way. Because of this, the authors noted that the MCMC chain tends to reach only locally optimal solutions. This is why Bayesian rule-based methods \citep{LethamRuMcMa15,YangRuSe16} have tended to be more successful in escaping local minima. This work builds specifically on that of \citet{YangRuSe16}. In particular, we use their fast bit-vector manipulations, and build on their bounds. 

\textit{Rule learning methods:} 
Most rule learning methods are not designed for optimality or interpretability, but mainly for computational speed and/or accuracy. In \textit{associative classification}, classifiers are formed either greedily from the top down as rule lists, \cite{Vanhoof10,Liu98,Li01,Yin03} or they are formed by taking the simple union of pre-mined rules \cite{??}, whereby any observation that fits into any of the rules is classified as positive \cite{??}. Inductive Logic Programming \cite{muggleton1994inductive} algorithms form disjunctive normal form patterns via a set of operations (rather than using optimization). These approaches are not appropriate for obtaining a guarantee of optimality. Methods for decision list learning construct rule lists iteratively in a greedy way
\cite{Rivest87,Sokolova03,Anthony05,Marchand05,RudinLeMa13,Goessling2015}, which again have no guarantee on optimality, and tend not to produce optimal rule lists in general. Some methods allow for interpretations of single rules, without constructing rule lists \citep{McCormick:2011ws}.

There is a tremendous amount of related work in other subfields that are too numerous to discuss at length here. We have not discussed \textit{rule mining} algorithms since they are part of an interchangeable preprocessing step for our algorithm, and are deterministically fast (that is, they will not generally slow our algorithm down). We also did not discuss methods that create disjunctive normal form models, e.g. logical analysis of data, and many associative classification methods). 

There are \textit{related problems concerning interpretable lists of rules.}
Rule lists of various flavors have been developed recently such as Falling Rule Lists \cite{WangRu}, which are constrained, as well as rule lists for dynamic treatment regimes \cite{ZhangEtAl15} and cost-sensitive dynamic treatment regimes \cite{LakkarajuRu17}. Both \cite{WangRu} and \cite{LakkarajuRu17} use Monte Carlo searches through the space of rule lists. The method proposed in this work could potentially be adapted to handle these kinds of interesting problems. We are currently working on bounds for Falling Rule Lists \cite{ChenRu17} similar to those presented here. 




%	Recent work in the field of decision lists has focused on the creation of probabilistic decision lists that generate a posterior distribution over the space of potential decision lists\citep{LethamRuMcMa15,YangRuSe16}. These methods achieve good accuracy while maintaining a small execution time. In addition, these methods improve on existing methods such as CART or C5.0 by optimizing over the global space of decision lists as opposed to searching for rules greedily and getting stuck at local optima. We take the same approach towards optimizing over the global search space, though we don’t use probabilistic techniques. In addition, we use the rule mining framework from \citep{LethamRuMcMa15} to generate the rules for our data sets. \citep{YangRuSe16} builds on \citep{LethamRuMcMa15} by placing bounds on the search space and creating a high performance bit vector manipulation library. We use that bit vector manipulation library to perform our computations, and add additional bounds to further prune the search space.




%Efficient Algorithms for Constructing Decision Trees with Constraints, Scalable Data Mining with Model Constraints, Building Decision Trees with Constraints

%	Our use of a branch and bound technique has also been applied to decision tree generation methods. \citep{garofalakis:2000-kdd} created an algorithm to generate more interpretable decision trees by allowing one to constrain the size of the decision tree. \citep{garofalakis:2000-kdd} uses branch-and-bound to constrain the size of the search space and limit the eventual size of the decision tree. During tree construction, \citep{garofalakis:2000-kdd} bounds the possible MDL cost of every different split at a given node. If every split at that node is more expensive than the actual cost of the current subtree, then that node can be pruned. In this way, they were able to prune the tree while constructing it instead of just constructing the tree and then pruning at the end.

%ProPublica
%	Certain problems require that the model used to solve that problem be interpretable as well as accurate. \citep{LarsonMaKiAn16} examines the problem of predicting recidivism and shows that a black box model, specifically the COMPAS score from the company Northpointe, has racially biased prediction. Black defendants are misclassified at a higher risk for recidivism than in actuality, while white defendants are misclassified at a lower risk. The model which produces the COMPAS scores is a black box algorithm which is not interpretable, and therefore the model does not provide a way for human input to correct for these racial biases. Our model produces similar accuracies to the logistic regression and COMPAS scores from \citep{LarsonMaKiAn16} while maintaining its interpretability.

\bibliographystyle{abbrvnat}
\bibliography{refs}

\end{document}

\section{Implementation}

Our algorithm is implemented with the use of a trie, a symmetry-aware map, and a queue as our core data structures. We use the trie as a cache to keep track of rule lists we have already evaluated. Each node in the trie contains the metadata associated with that corresponding rule list. This metadata includes bookkeeping information such as what child rule lists are feasible as well as information such as the lower bound and prediction for that rule list. Our trie also tracks the best observed rule list by keeping track of the minimum objective found and its associated rule list.

We implement the symmetry-aware map as an STL unordered\_map. We have two different versions of the map that both allow identical permutations to map to the same key. In both cases, the values contain the actual ordering of the rules that represents the current best prefix for that permutation. In the first version, keys to the map are the canonical order of rule lists: i.e. the lists 3-2-1 and 2-3-1 both map to 1-2-3. The second version has keys that represent the captured data points. These are equivalent for the rule lists 3-2-1 and 2-3-1, so both will map to the same key. Most data sets have a large amount of data, so the keys representing the canonical ordering are usually more efficient. In general, for long runs of our algorithm, the symmetry-aware map dominates our memory usage. When inserting into the symmetry-aware map, we check if there is a permutation P1 of this rule list P0 already in the map. If the objective of P0 is better than the objective of P1, we update the map and remove P1 and its subtree from the trie. Otherwise we don't insert P0 into the symmetry-aware map or the trie.

We use a queue to index all of the leaves of the trie that still need to be explored. We implement a number of different scheduling schemes including BFS, DFS, and a priority queue. We also have a stochastic exploration process that bypasses the use of a queue by walking down the trie, randomly choosing a child each time, until a leaf is chosen to be explored. The priority queue allows us to order our exploration by different metrics such as the lower bound, the objective, or the curiosity of the leaves. We find that ordering by curiosity will often, but not always, lead to a faster runtime than using BFS.

Our program executes as follows. While there are still leaves of the trie to be explored, we use our scheduling policy to select the next rule list to evaluate. Then, for every rule that is not already in this rule list, we calculate its lower bound, objective, and other metrics that would occur if the rule were added to the end of the rule list. If our new rule list has a lower bound that is below the minimum objective, then we insert it into the symmetry-aware map, the tree, and the queue and update the minimum objective if necessary. If the lower bound is greater than the minimum objective, meaning this new rule list could never be better than one we have already seen, we don't insert the new rule list into the tree or queue. 

Every time we update the minimum objective, we garbage collect the trie. We do this by traversing from the root the all of the leaves, deleting any subtrees of nodes with a lower bound that is larger than the minimum objective. In addition, if we encounter a node with no children, we prune upwards--deleting that node and recursively traversing the tree towards the root, deleting any childless nodes. This garbage collection allows us to limit the memory usage of the trie, though in our experiments we observe the minimum objective to decrease only a small number of times.

%\section{Introduction}

The \textit{proceedings} are the records of a conference\footnote{This
  is a footnote}.  ACM seeks
to give these conference by-products a uniform, high-quality
appearance.  To do this, ACM has some rigid requirements for the
format of the proceedings documents: there is a specified format
(balanced double columns), a specified set of fonts (Arial or
Helvetica and Times Roman) in certain specified sizes, a specified
live area, centered on the page, specified size of margins, specified
column width and gutter size.

\section{The Body of The Paper}
Typically, the body of a paper is organized into a hierarchical
structure, with numbered or unnumbered headings for sections,
subsections, sub-subsections, and even smaller sections.  The command
\texttt{{\char'134}section} that precedes this paragraph is part of
such a hierarchy.\footnote{This is a footnote.} \LaTeX\ handles the
numbering and placement of these headings for you, when you use the
appropriate heading commands around the titles of the headings.  If
you want a sub-subsection or smaller part to be unnumbered in your
output, simply append an asterisk to the command name.  Examples of
both numbered and unnumbered headings will appear throughout the
balance of this sample document.

Because the entire article is contained in the \textbf{document}
environment, you can indicate the start of a new paragraph with a
blank line in your input file; that is why this sentence forms a
separate paragraph.

\subsection{Type Changes and {\itshape Special} Characters}

We have already seen several typeface changes in this sample.  You can
indicate italicized words or phrases in your text with the command
\texttt{{\char'134}textit}; emboldening with the command
\texttt{{\char'134}textbf} and typewriter-style (for instance, for
computer code) with \texttt{{\char'134}texttt}.  But remember, you do
not have to indicate typestyle changes when such changes are part of
the \textit{structural} elements of your article; for instance, the
heading of this subsection will be in a sans serif\footnote{Another
  footnote, here.  Let's make this a rather short one to see how it
  looks.} typeface, but that is handled by the document class file.
Take care with the use of\footnote{A third, and last, footnote.}  the
curly braces in typeface changes; they mark the beginning and end of
the text that is to be in the different typeface.

You can use whatever symbols, accented characters, or non-English
characters you need anywhere in your document; you can find a complete
list of what is available in the \textit{\LaTeX\ User's Guide}
\cite{Lamport:LaTeX}.

\subsection{Math Equations}
You may want to display math equations in three distinct styles:
inline, numbered or non-numbered display.  Each of
the three are discussed in the next sections.

\subsubsection{Inline (In-text) Equations}
A formula that appears in the running text is called an
inline or in-text formula.  It is produced by the
\textbf{math} environment, which can be
invoked with the usual \texttt{{\char'134}begin\,\ldots{\char'134}end}
construction or with the short form \texttt{\$\,\ldots\$}. You
can use any of the symbols and structures,
from $\alpha$ to $\omega$, available in
\LaTeX~\cite{Lamport:LaTeX}; this section will simply show a
few examples of in-text equations in context. Notice how
this equation:
\begin{math}
  \lim_{n\rightarrow \infty}x=0
\end{math},
set here in in-line math style, looks slightly different when
set in display style.  (See next section).

\subsubsection{Display Equations}
A numbered display equation---one set off by vertical space from the
text and centered horizontally---is produced by the \textbf{equation}
environment. An unnumbered display equation is produced by the
\textbf{displaymath} environment.

Again, in either environment, you can use any of the symbols
and structures available in \LaTeX\@; this section will just
give a couple of examples of display equations in context.
First, consider the equation, shown as an inline equation above:
\begin{equation}
  \lim_{n\rightarrow \infty}x=0
\end{equation}
Notice how it is formatted somewhat differently in
the \textbf{displaymath}
environment.  Now, we'll enter an unnumbered equation:
\begin{displaymath}
  \sum_{i=0}^{\infty} x + 1
\end{displaymath}
and follow it with another numbered equation:
\begin{equation}
  \sum_{i=0}^{\infty}x_i=\int_{0}^{\pi+2} f
\end{equation}
just to demonstrate \LaTeX's able handling of numbering.

\subsection{Citations}
Citations to articles~\cite{bowman:reasoning,
clark:pct, braams:babel, herlihy:methodology},
conference proceedings~\cite{clark:pct} or maybe
books \cite{Lamport:LaTeX, salas:calculus} listed
in the Bibliography section of your
article will occur throughout the text of your article.
You should use BibTeX to automatically produce this bibliography;
you simply need to insert one of several citation commands with
a key of the item cited in the proper location in
the \texttt{.tex} file~\cite{Lamport:LaTeX}.
The key is a short reference you invent to uniquely
identify each work; in this sample document, the key is
the first author's surname and a
word from the title.  This identifying key is included
with each item in the \texttt{.bib} file for your article.

The details of the construction of the \texttt{.bib} file
are beyond the scope of this sample document, but more
information can be found in the \textit{Author's Guide},
and exhaustive details in the \textit{\LaTeX\ User's
Guide} by Lamport~\shortcite{Lamport:LaTeX}.


This article shows only the plainest form
of the citation command, using \texttt{{\char'134}cite}.

\subsection{Tables}
Because tables cannot be split across pages, the best
placement for them is typically the top of the page
nearest their initial cite.  To
ensure this proper ``floating'' placement of tables, use the
environment \textbf{table} to enclose the table's contents and
the table caption.  The contents of the table itself must go
in the \textbf{tabular} environment, to
be aligned properly in rows and columns, with the desired
horizontal and vertical rules.  Again, detailed instructions
on \textbf{tabular} material
are found in the \textit{\LaTeX\ User's Guide}.

Immediately following this sentence is the point at which
Table~\ref{tab:freq} is included in the input file; compare the
placement of the table here with the table in the printed
output of this document.

\begin{table}
  \caption{Frequency of Special Characters}
  \label{tab:freq}
  \begin{tabular}{ccl}
    \toprule
    Non-English or Math&Frequency&Comments\\
    \midrule
    \O & 1 in 1,000& For Swedish names\\
    $\pi$ & 1 in 5& Common in math\\
    \$ & 4 in 5 & Used in business\\
    $\Psi^2_1$ & 1 in 40,000& Unexplained usage\\
  \bottomrule
\end{tabular}
\end{table}

To set a wider table, which takes up the whole width of the page's
live area, use the environment \textbf{table*} to enclose the table's
contents and the table caption.  As with a single-column table, this
wide table will ``float'' to a location deemed more desirable.
Immediately following this sentence is the point at which
Table~\ref{tab:commands} is included in the input file; again, it is
instructive to compare the placement of the table here with the table
in the printed output of this document.


\begin{table*}
  \caption{Some Typical Commands}
  \label{tab:commands}
  \begin{tabular}{ccl}
    \toprule
    Command &A Number & Comments\\
    \midrule
    \texttt{{\char'134}author} & 100& Author \\
    \texttt{{\char'134}table}& 300 & For tables\\
    \texttt{{\char'134}table*}& 400& For wider tables\\
    \bottomrule
  \end{tabular}
\end{table*}
% end the environment with {table*}, NOTE not {table}!

It is strongly recommended to use the package booktabs~\cite{Fear05}
and follow its main principles of typography with respect to tables:
\begin{enumerate}
\item Never, ever use vertical rules.
\item Never use double rules.
\end{enumerate}
It is also a good idea not to overuse horizontal rules.


\subsection{Figures}

Like tables, figures cannot be split across pages; the best placement
for them is typically the top or the bottom of the page nearest their
initial cite.  To ensure this proper ``floating'' placement of
figures, use the environment \textbf{figure} to enclose the figure and
its caption.

This sample document contains examples of \texttt{.eps} files to be
displayable with \LaTeX.  If you work with pdf\LaTeX, use files in the
\texttt{.pdf} format.  Note that most modern \TeX\ systems will convert
\texttt{.eps} to \texttt{.pdf} for you on the fly.  More details on
each of these are found in the \textit{Author's Guide}.

\begin{figure}
\includegraphics{fly}
\caption{A sample black and white graphic.}
\end{figure}

\begin{figure}
\includegraphics[height=1in, width=1in]{fly}
\caption{A sample black and white graphic
that has been resized with the \texttt{includegraphics} command.}
\end{figure}


As was the case with tables, you may want a figure that spans two
columns.  To do this, and still to ensure proper ``floating''
placement of tables, use the environment \textbf{figure*} to enclose
the figure and its caption.  And don't forget to end the environment
with \textbf{figure*}, not \textbf{figure}!

\begin{figure*}
\includegraphics{flies}
\caption{A sample black and white graphic
that needs to span two columns of text.}
\end{figure*}


\begin{figure}
\includegraphics[height=1in, width=1in]{rosette}
\caption{A sample black and white graphic that has
been resized with the \texttt{includegraphics} command.}
\end{figure}

\subsection{Theorem-like Constructs}

Other common constructs that may occur in your article are the forms
for logical constructs like theorems, axioms, corollaries and proofs.
ACM uses two types of these constructs:  theorem-like and
definition-like.

Here is a theorem:
\begin{theorem}
  Let $f$ be continuous on $[a,b]$.  If $G$ is
  an antiderivative for $f$ on $[a,b]$, then
  \begin{displaymath}
    \int^b_af(t)\,dt = G(b) - G(a).
  \end{displaymath}
\end{theorem}

Here is a definition:
\begin{definition}
  If $z$ is irrational, then by $e^z$ we mean the
  unique number that has
  logarithm $z$:
  \begin{displaymath}
    \log e^z = z.
  \end{displaymath}
\end{definition}

The pre-defined theorem-like constructs are \textbf{theorem},
\textbf{conjecture}, \textbf{proposition}, \textbf{lemma} and
\textbf{corollary}.  The pre-defined de\-fi\-ni\-ti\-on-like constructs are
\textbf{example} and \textbf{definition}.  You can add your own
constructs using the \textsl{amsthm} interface~\cite{Amsthm15}.  The
styles used in the \verb|\theoremstyle| command are \textbf{acmplain}
and \textbf{acmdefinition}.

Another construct is \textbf{proof}, for example,

\begin{proof}
  Suppose on the contrary there exists a real number $L$ such that
  \begin{displaymath}
    \lim_{x\rightarrow\infty} \frac{f(x)}{g(x)} = L.
  \end{displaymath}
  Then
  \begin{displaymath}
    l=\lim_{x\rightarrow c} f(x)
    = \lim_{x\rightarrow c}
    \left[ g{x} \cdot \frac{f(x)}{g(x)} \right ]
    = \lim_{x\rightarrow c} g(x) \cdot \lim_{x\rightarrow c}
    \frac{f(x)}{g(x)} = 0\cdot L = 0,
  \end{displaymath}
  which contradicts our assumption that $l\neq 0$.
\end{proof}

\section{Conclusions}
This paragraph will end the body of this sample document.
Remember that you might still have Acknowledgments or
Appendices; brief samples of these
follow.  There is still the Bibliography to deal with; and
we will make a disclaimer about that here: with the exception
of the reference to the \LaTeX\ book, the citations in
this paper are to articles which have nothing to
do with the present subject and are used as
examples only.
%\end{document}  % This is where a 'short' article might terminate



\appendix
%Appendix A
\section{Headings in Appendices}
The rules about hierarchical headings discussed above for
the body of the article are different in the appendices.
In the \textbf{appendix} environment, the command
\textbf{section} is used to
indicate the start of each Appendix, with alphabetic order
designation (i.e., the first is A, the second B, etc.) and
a title (if you include one).  So, if you need
hierarchical structure
\textit{within} an Appendix, start with \textbf{subsection} as the
highest level. Here is an outline of the body of this
document in Appendix-appropriate form:
\subsection{Introduction}
\subsection{The Body of the Paper}
\subsubsection{Type Changes and  Special Characters}
\subsubsection{Math Equations}
\paragraph{Inline (In-text) Equations}
\paragraph{Display Equations}
\subsubsection{Citations}
\subsubsection{Tables}
\subsubsection{Figures}
\subsubsection{Theorem-like Constructs}
\subsubsection*{A Caveat for the \TeX\ Expert}
\subsection{Conclusions}
\subsection{References}
Generated by bibtex from your \texttt{.bib} file.  Run latex,
then bibtex, then latex twice (to resolve references)
to create the \texttt{.bbl} file.  Insert that \texttt{.bbl}
file into the \texttt{.tex} source file and comment out
the command \texttt{{\char'134}thebibliography}.
% This next section command marks the start of
% Appendix B, and does not continue the present hierarchy
\section{More Help for the Hardy}

Of course, reading the source code is always useful.  The file
\path{acmart.pdf} contains both the user guide and the commented
code.

\begin{acks}
  The authors would like to thank Dr. Yuhua Li for providing the
  matlab code of  the \textit{BEPS} method. 

  The authors would also like to thank the anonymous referees for
  their valuable comments and helpful suggestions. The work is
  supported by the \grantsponsor{GS501100001809}{National Natural
    Science Foundation of
    China}{http://dx.doi.org/10.13039/501100001809} under Grant
  No.:~\grantnum{GS501100001809}{61273304}
  and~\grantnum[http://www.nnsf.cn/youngscientsts]{GS501100001809}{Young
    Scientsts' Support Program}.

\end{acks}


\bibliographystyle{ACM-Reference-Format}
\bibliography{../paper/refs} 

\end{document}
